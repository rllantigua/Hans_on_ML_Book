\documentclass[11pt]{article}

    \usepackage[breakable]{tcolorbox}
    \usepackage{parskip} % Stop auto-indenting (to mimic markdown behaviour)
    
    \usepackage{iftex}
    \ifPDFTeX
    	\usepackage[T1]{fontenc}
    	\usepackage{mathpazo}
    \else
    	\usepackage{fontspec}
    \fi

    % Basic figure setup, for now with no caption control since it's done
    % automatically by Pandoc (which extracts ![](path) syntax from Markdown).
    \usepackage{graphicx}
    % Maintain compatibility with old templates. Remove in nbconvert 6.0
    \let\Oldincludegraphics\includegraphics
    % Ensure that by default, figures have no caption (until we provide a
    % proper Figure object with a Caption API and a way to capture that
    % in the conversion process - todo).
    \usepackage{caption}
    \DeclareCaptionFormat{nocaption}{}
    \captionsetup{format=nocaption,aboveskip=0pt,belowskip=0pt}

    \usepackage[Export]{adjustbox} % Used to constrain images to a maximum size
    \adjustboxset{max size={0.9\linewidth}{0.9\paperheight}}
    \usepackage{float}
    \floatplacement{figure}{H} % forces figures to be placed at the correct location
    \usepackage{xcolor} % Allow colors to be defined
    \usepackage{enumerate} % Needed for markdown enumerations to work
    \usepackage{geometry} % Used to adjust the document margins
    \usepackage{amsmath} % Equations
    \usepackage{amssymb} % Equations
    \usepackage{textcomp} % defines textquotesingle
    % Hack from http://tex.stackexchange.com/a/47451/13684:
    \AtBeginDocument{%
        \def\PYZsq{\textquotesingle}% Upright quotes in Pygmentized code
    }
    \usepackage{upquote} % Upright quotes for verbatim code
    \usepackage{eurosym} % defines \euro
    \usepackage[mathletters]{ucs} % Extended unicode (utf-8) support
    \usepackage{fancyvrb} % verbatim replacement that allows latex
    \usepackage{grffile} % extends the file name processing of package graphics 
                         % to support a larger range
    \makeatletter % fix for grffile with XeLaTeX
    \def\Gread@@xetex#1{%
      \IfFileExists{"\Gin@base".bb}%
      {\Gread@eps{\Gin@base.bb}}%
      {\Gread@@xetex@aux#1}%
    }
    \makeatother

    % The hyperref package gives us a pdf with properly built
    % internal navigation ('pdf bookmarks' for the table of contents,
    % internal cross-reference links, web links for URLs, etc.)
    \usepackage{hyperref}
    % The default LaTeX title has an obnoxious amount of whitespace. By default,
    % titling removes some of it. It also provides customization options.
    \usepackage{titling}
    \usepackage{longtable} % longtable support required by pandoc >1.10
    \usepackage{booktabs}  % table support for pandoc > 1.12.2
    \usepackage[inline]{enumitem} % IRkernel/repr support (it uses the enumerate* environment)
    \usepackage[normalem]{ulem} % ulem is needed to support strikethroughs (\sout)
                                % normalem makes italics be italics, not underlines
    \usepackage{mathrsfs}
    

    
    % Colors for the hyperref package
    \definecolor{urlcolor}{rgb}{0,.145,.698}
    \definecolor{linkcolor}{rgb}{.71,0.21,0.01}
    \definecolor{citecolor}{rgb}{.12,.54,.11}

    % ANSI colors
    \definecolor{ansi-black}{HTML}{3E424D}
    \definecolor{ansi-black-intense}{HTML}{282C36}
    \definecolor{ansi-red}{HTML}{E75C58}
    \definecolor{ansi-red-intense}{HTML}{B22B31}
    \definecolor{ansi-green}{HTML}{00A250}
    \definecolor{ansi-green-intense}{HTML}{007427}
    \definecolor{ansi-yellow}{HTML}{DDB62B}
    \definecolor{ansi-yellow-intense}{HTML}{B27D12}
    \definecolor{ansi-blue}{HTML}{208FFB}
    \definecolor{ansi-blue-intense}{HTML}{0065CA}
    \definecolor{ansi-magenta}{HTML}{D160C4}
    \definecolor{ansi-magenta-intense}{HTML}{A03196}
    \definecolor{ansi-cyan}{HTML}{60C6C8}
    \definecolor{ansi-cyan-intense}{HTML}{258F8F}
    \definecolor{ansi-white}{HTML}{C5C1B4}
    \definecolor{ansi-white-intense}{HTML}{A1A6B2}
    \definecolor{ansi-default-inverse-fg}{HTML}{FFFFFF}
    \definecolor{ansi-default-inverse-bg}{HTML}{000000}

    % commands and environments needed by pandoc snippets
    % extracted from the output of `pandoc -s`
    \providecommand{\tightlist}{%
      \setlength{\itemsep}{0pt}\setlength{\parskip}{0pt}}
    \DefineVerbatimEnvironment{Highlighting}{Verbatim}{commandchars=\\\{\}}
    % Add ',fontsize=\small' for more characters per line
    \newenvironment{Shaded}{}{}
    \newcommand{\KeywordTok}[1]{\textcolor[rgb]{0.00,0.44,0.13}{\textbf{{#1}}}}
    \newcommand{\DataTypeTok}[1]{\textcolor[rgb]{0.56,0.13,0.00}{{#1}}}
    \newcommand{\DecValTok}[1]{\textcolor[rgb]{0.25,0.63,0.44}{{#1}}}
    \newcommand{\BaseNTok}[1]{\textcolor[rgb]{0.25,0.63,0.44}{{#1}}}
    \newcommand{\FloatTok}[1]{\textcolor[rgb]{0.25,0.63,0.44}{{#1}}}
    \newcommand{\CharTok}[1]{\textcolor[rgb]{0.25,0.44,0.63}{{#1}}}
    \newcommand{\StringTok}[1]{\textcolor[rgb]{0.25,0.44,0.63}{{#1}}}
    \newcommand{\CommentTok}[1]{\textcolor[rgb]{0.38,0.63,0.69}{\textit{{#1}}}}
    \newcommand{\OtherTok}[1]{\textcolor[rgb]{0.00,0.44,0.13}{{#1}}}
    \newcommand{\AlertTok}[1]{\textcolor[rgb]{1.00,0.00,0.00}{\textbf{{#1}}}}
    \newcommand{\FunctionTok}[1]{\textcolor[rgb]{0.02,0.16,0.49}{{#1}}}
    \newcommand{\RegionMarkerTok}[1]{{#1}}
    \newcommand{\ErrorTok}[1]{\textcolor[rgb]{1.00,0.00,0.00}{\textbf{{#1}}}}
    \newcommand{\NormalTok}[1]{{#1}}
    
    % Additional commands for more recent versions of Pandoc
    \newcommand{\ConstantTok}[1]{\textcolor[rgb]{0.53,0.00,0.00}{{#1}}}
    \newcommand{\SpecialCharTok}[1]{\textcolor[rgb]{0.25,0.44,0.63}{{#1}}}
    \newcommand{\VerbatimStringTok}[1]{\textcolor[rgb]{0.25,0.44,0.63}{{#1}}}
    \newcommand{\SpecialStringTok}[1]{\textcolor[rgb]{0.73,0.40,0.53}{{#1}}}
    \newcommand{\ImportTok}[1]{{#1}}
    \newcommand{\DocumentationTok}[1]{\textcolor[rgb]{0.73,0.13,0.13}{\textit{{#1}}}}
    \newcommand{\AnnotationTok}[1]{\textcolor[rgb]{0.38,0.63,0.69}{\textbf{\textit{{#1}}}}}
    \newcommand{\CommentVarTok}[1]{\textcolor[rgb]{0.38,0.63,0.69}{\textbf{\textit{{#1}}}}}
    \newcommand{\VariableTok}[1]{\textcolor[rgb]{0.10,0.09,0.49}{{#1}}}
    \newcommand{\ControlFlowTok}[1]{\textcolor[rgb]{0.00,0.44,0.13}{\textbf{{#1}}}}
    \newcommand{\OperatorTok}[1]{\textcolor[rgb]{0.40,0.40,0.40}{{#1}}}
    \newcommand{\BuiltInTok}[1]{{#1}}
    \newcommand{\ExtensionTok}[1]{{#1}}
    \newcommand{\PreprocessorTok}[1]{\textcolor[rgb]{0.74,0.48,0.00}{{#1}}}
    \newcommand{\AttributeTok}[1]{\textcolor[rgb]{0.49,0.56,0.16}{{#1}}}
    \newcommand{\InformationTok}[1]{\textcolor[rgb]{0.38,0.63,0.69}{\textbf{\textit{{#1}}}}}
    \newcommand{\WarningTok}[1]{\textcolor[rgb]{0.38,0.63,0.69}{\textbf{\textit{{#1}}}}}
    
    
    % Define a nice break command that doesn't care if a line doesn't already
    % exist.
    \def\br{\hspace*{\fill} \\* }
    % Math Jax compatibility definitions
    \def\gt{>}
    \def\lt{<}
    \let\Oldtex\TeX
    \let\Oldlatex\LaTeX
    \renewcommand{\TeX}{\textrm{\Oldtex}}
    \renewcommand{\LaTeX}{\textrm{\Oldlatex}}
    % Document parameters
    % Document title
    \title{Cap\_2}
    
    
    
    
    
% Pygments definitions
\makeatletter
\def\PY@reset{\let\PY@it=\relax \let\PY@bf=\relax%
    \let\PY@ul=\relax \let\PY@tc=\relax%
    \let\PY@bc=\relax \let\PY@ff=\relax}
\def\PY@tok#1{\csname PY@tok@#1\endcsname}
\def\PY@toks#1+{\ifx\relax#1\empty\else%
    \PY@tok{#1}\expandafter\PY@toks\fi}
\def\PY@do#1{\PY@bc{\PY@tc{\PY@ul{%
    \PY@it{\PY@bf{\PY@ff{#1}}}}}}}
\def\PY#1#2{\PY@reset\PY@toks#1+\relax+\PY@do{#2}}

\expandafter\def\csname PY@tok@w\endcsname{\def\PY@tc##1{\textcolor[rgb]{0.73,0.73,0.73}{##1}}}
\expandafter\def\csname PY@tok@c\endcsname{\let\PY@it=\textit\def\PY@tc##1{\textcolor[rgb]{0.25,0.50,0.50}{##1}}}
\expandafter\def\csname PY@tok@cp\endcsname{\def\PY@tc##1{\textcolor[rgb]{0.74,0.48,0.00}{##1}}}
\expandafter\def\csname PY@tok@k\endcsname{\let\PY@bf=\textbf\def\PY@tc##1{\textcolor[rgb]{0.00,0.50,0.00}{##1}}}
\expandafter\def\csname PY@tok@kp\endcsname{\def\PY@tc##1{\textcolor[rgb]{0.00,0.50,0.00}{##1}}}
\expandafter\def\csname PY@tok@kt\endcsname{\def\PY@tc##1{\textcolor[rgb]{0.69,0.00,0.25}{##1}}}
\expandafter\def\csname PY@tok@o\endcsname{\def\PY@tc##1{\textcolor[rgb]{0.40,0.40,0.40}{##1}}}
\expandafter\def\csname PY@tok@ow\endcsname{\let\PY@bf=\textbf\def\PY@tc##1{\textcolor[rgb]{0.67,0.13,1.00}{##1}}}
\expandafter\def\csname PY@tok@nb\endcsname{\def\PY@tc##1{\textcolor[rgb]{0.00,0.50,0.00}{##1}}}
\expandafter\def\csname PY@tok@nf\endcsname{\def\PY@tc##1{\textcolor[rgb]{0.00,0.00,1.00}{##1}}}
\expandafter\def\csname PY@tok@nc\endcsname{\let\PY@bf=\textbf\def\PY@tc##1{\textcolor[rgb]{0.00,0.00,1.00}{##1}}}
\expandafter\def\csname PY@tok@nn\endcsname{\let\PY@bf=\textbf\def\PY@tc##1{\textcolor[rgb]{0.00,0.00,1.00}{##1}}}
\expandafter\def\csname PY@tok@ne\endcsname{\let\PY@bf=\textbf\def\PY@tc##1{\textcolor[rgb]{0.82,0.25,0.23}{##1}}}
\expandafter\def\csname PY@tok@nv\endcsname{\def\PY@tc##1{\textcolor[rgb]{0.10,0.09,0.49}{##1}}}
\expandafter\def\csname PY@tok@no\endcsname{\def\PY@tc##1{\textcolor[rgb]{0.53,0.00,0.00}{##1}}}
\expandafter\def\csname PY@tok@nl\endcsname{\def\PY@tc##1{\textcolor[rgb]{0.63,0.63,0.00}{##1}}}
\expandafter\def\csname PY@tok@ni\endcsname{\let\PY@bf=\textbf\def\PY@tc##1{\textcolor[rgb]{0.60,0.60,0.60}{##1}}}
\expandafter\def\csname PY@tok@na\endcsname{\def\PY@tc##1{\textcolor[rgb]{0.49,0.56,0.16}{##1}}}
\expandafter\def\csname PY@tok@nt\endcsname{\let\PY@bf=\textbf\def\PY@tc##1{\textcolor[rgb]{0.00,0.50,0.00}{##1}}}
\expandafter\def\csname PY@tok@nd\endcsname{\def\PY@tc##1{\textcolor[rgb]{0.67,0.13,1.00}{##1}}}
\expandafter\def\csname PY@tok@s\endcsname{\def\PY@tc##1{\textcolor[rgb]{0.73,0.13,0.13}{##1}}}
\expandafter\def\csname PY@tok@sd\endcsname{\let\PY@it=\textit\def\PY@tc##1{\textcolor[rgb]{0.73,0.13,0.13}{##1}}}
\expandafter\def\csname PY@tok@si\endcsname{\let\PY@bf=\textbf\def\PY@tc##1{\textcolor[rgb]{0.73,0.40,0.53}{##1}}}
\expandafter\def\csname PY@tok@se\endcsname{\let\PY@bf=\textbf\def\PY@tc##1{\textcolor[rgb]{0.73,0.40,0.13}{##1}}}
\expandafter\def\csname PY@tok@sr\endcsname{\def\PY@tc##1{\textcolor[rgb]{0.73,0.40,0.53}{##1}}}
\expandafter\def\csname PY@tok@ss\endcsname{\def\PY@tc##1{\textcolor[rgb]{0.10,0.09,0.49}{##1}}}
\expandafter\def\csname PY@tok@sx\endcsname{\def\PY@tc##1{\textcolor[rgb]{0.00,0.50,0.00}{##1}}}
\expandafter\def\csname PY@tok@m\endcsname{\def\PY@tc##1{\textcolor[rgb]{0.40,0.40,0.40}{##1}}}
\expandafter\def\csname PY@tok@gh\endcsname{\let\PY@bf=\textbf\def\PY@tc##1{\textcolor[rgb]{0.00,0.00,0.50}{##1}}}
\expandafter\def\csname PY@tok@gu\endcsname{\let\PY@bf=\textbf\def\PY@tc##1{\textcolor[rgb]{0.50,0.00,0.50}{##1}}}
\expandafter\def\csname PY@tok@gd\endcsname{\def\PY@tc##1{\textcolor[rgb]{0.63,0.00,0.00}{##1}}}
\expandafter\def\csname PY@tok@gi\endcsname{\def\PY@tc##1{\textcolor[rgb]{0.00,0.63,0.00}{##1}}}
\expandafter\def\csname PY@tok@gr\endcsname{\def\PY@tc##1{\textcolor[rgb]{1.00,0.00,0.00}{##1}}}
\expandafter\def\csname PY@tok@ge\endcsname{\let\PY@it=\textit}
\expandafter\def\csname PY@tok@gs\endcsname{\let\PY@bf=\textbf}
\expandafter\def\csname PY@tok@gp\endcsname{\let\PY@bf=\textbf\def\PY@tc##1{\textcolor[rgb]{0.00,0.00,0.50}{##1}}}
\expandafter\def\csname PY@tok@go\endcsname{\def\PY@tc##1{\textcolor[rgb]{0.53,0.53,0.53}{##1}}}
\expandafter\def\csname PY@tok@gt\endcsname{\def\PY@tc##1{\textcolor[rgb]{0.00,0.27,0.87}{##1}}}
\expandafter\def\csname PY@tok@err\endcsname{\def\PY@bc##1{\setlength{\fboxsep}{0pt}\fcolorbox[rgb]{1.00,0.00,0.00}{1,1,1}{\strut ##1}}}
\expandafter\def\csname PY@tok@kc\endcsname{\let\PY@bf=\textbf\def\PY@tc##1{\textcolor[rgb]{0.00,0.50,0.00}{##1}}}
\expandafter\def\csname PY@tok@kd\endcsname{\let\PY@bf=\textbf\def\PY@tc##1{\textcolor[rgb]{0.00,0.50,0.00}{##1}}}
\expandafter\def\csname PY@tok@kn\endcsname{\let\PY@bf=\textbf\def\PY@tc##1{\textcolor[rgb]{0.00,0.50,0.00}{##1}}}
\expandafter\def\csname PY@tok@kr\endcsname{\let\PY@bf=\textbf\def\PY@tc##1{\textcolor[rgb]{0.00,0.50,0.00}{##1}}}
\expandafter\def\csname PY@tok@bp\endcsname{\def\PY@tc##1{\textcolor[rgb]{0.00,0.50,0.00}{##1}}}
\expandafter\def\csname PY@tok@fm\endcsname{\def\PY@tc##1{\textcolor[rgb]{0.00,0.00,1.00}{##1}}}
\expandafter\def\csname PY@tok@vc\endcsname{\def\PY@tc##1{\textcolor[rgb]{0.10,0.09,0.49}{##1}}}
\expandafter\def\csname PY@tok@vg\endcsname{\def\PY@tc##1{\textcolor[rgb]{0.10,0.09,0.49}{##1}}}
\expandafter\def\csname PY@tok@vi\endcsname{\def\PY@tc##1{\textcolor[rgb]{0.10,0.09,0.49}{##1}}}
\expandafter\def\csname PY@tok@vm\endcsname{\def\PY@tc##1{\textcolor[rgb]{0.10,0.09,0.49}{##1}}}
\expandafter\def\csname PY@tok@sa\endcsname{\def\PY@tc##1{\textcolor[rgb]{0.73,0.13,0.13}{##1}}}
\expandafter\def\csname PY@tok@sb\endcsname{\def\PY@tc##1{\textcolor[rgb]{0.73,0.13,0.13}{##1}}}
\expandafter\def\csname PY@tok@sc\endcsname{\def\PY@tc##1{\textcolor[rgb]{0.73,0.13,0.13}{##1}}}
\expandafter\def\csname PY@tok@dl\endcsname{\def\PY@tc##1{\textcolor[rgb]{0.73,0.13,0.13}{##1}}}
\expandafter\def\csname PY@tok@s2\endcsname{\def\PY@tc##1{\textcolor[rgb]{0.73,0.13,0.13}{##1}}}
\expandafter\def\csname PY@tok@sh\endcsname{\def\PY@tc##1{\textcolor[rgb]{0.73,0.13,0.13}{##1}}}
\expandafter\def\csname PY@tok@s1\endcsname{\def\PY@tc##1{\textcolor[rgb]{0.73,0.13,0.13}{##1}}}
\expandafter\def\csname PY@tok@mb\endcsname{\def\PY@tc##1{\textcolor[rgb]{0.40,0.40,0.40}{##1}}}
\expandafter\def\csname PY@tok@mf\endcsname{\def\PY@tc##1{\textcolor[rgb]{0.40,0.40,0.40}{##1}}}
\expandafter\def\csname PY@tok@mh\endcsname{\def\PY@tc##1{\textcolor[rgb]{0.40,0.40,0.40}{##1}}}
\expandafter\def\csname PY@tok@mi\endcsname{\def\PY@tc##1{\textcolor[rgb]{0.40,0.40,0.40}{##1}}}
\expandafter\def\csname PY@tok@il\endcsname{\def\PY@tc##1{\textcolor[rgb]{0.40,0.40,0.40}{##1}}}
\expandafter\def\csname PY@tok@mo\endcsname{\def\PY@tc##1{\textcolor[rgb]{0.40,0.40,0.40}{##1}}}
\expandafter\def\csname PY@tok@ch\endcsname{\let\PY@it=\textit\def\PY@tc##1{\textcolor[rgb]{0.25,0.50,0.50}{##1}}}
\expandafter\def\csname PY@tok@cm\endcsname{\let\PY@it=\textit\def\PY@tc##1{\textcolor[rgb]{0.25,0.50,0.50}{##1}}}
\expandafter\def\csname PY@tok@cpf\endcsname{\let\PY@it=\textit\def\PY@tc##1{\textcolor[rgb]{0.25,0.50,0.50}{##1}}}
\expandafter\def\csname PY@tok@c1\endcsname{\let\PY@it=\textit\def\PY@tc##1{\textcolor[rgb]{0.25,0.50,0.50}{##1}}}
\expandafter\def\csname PY@tok@cs\endcsname{\let\PY@it=\textit\def\PY@tc##1{\textcolor[rgb]{0.25,0.50,0.50}{##1}}}

\def\PYZbs{\char`\\}
\def\PYZus{\char`\_}
\def\PYZob{\char`\{}
\def\PYZcb{\char`\}}
\def\PYZca{\char`\^}
\def\PYZam{\char`\&}
\def\PYZlt{\char`\<}
\def\PYZgt{\char`\>}
\def\PYZsh{\char`\#}
\def\PYZpc{\char`\%}
\def\PYZdl{\char`\$}
\def\PYZhy{\char`\-}
\def\PYZsq{\char`\'}
\def\PYZdq{\char`\"}
\def\PYZti{\char`\~}
% for compatibility with earlier versions
\def\PYZat{@}
\def\PYZlb{[}
\def\PYZrb{]}
\makeatother


    % For linebreaks inside Verbatim environment from package fancyvrb. 
    \makeatletter
        \newbox\Wrappedcontinuationbox 
        \newbox\Wrappedvisiblespacebox 
        \newcommand*\Wrappedvisiblespace {\textcolor{red}{\textvisiblespace}} 
        \newcommand*\Wrappedcontinuationsymbol {\textcolor{red}{\llap{\tiny$\m@th\hookrightarrow$}}} 
        \newcommand*\Wrappedcontinuationindent {3ex } 
        \newcommand*\Wrappedafterbreak {\kern\Wrappedcontinuationindent\copy\Wrappedcontinuationbox} 
        % Take advantage of the already applied Pygments mark-up to insert 
        % potential linebreaks for TeX processing. 
        %        {, <, #, %, $, ' and ": go to next line. 
        %        _, }, ^, &, >, - and ~: stay at end of broken line. 
        % Use of \textquotesingle for straight quote. 
        \newcommand*\Wrappedbreaksatspecials {% 
            \def\PYGZus{\discretionary{\char`\_}{\Wrappedafterbreak}{\char`\_}}% 
            \def\PYGZob{\discretionary{}{\Wrappedafterbreak\char`\{}{\char`\{}}% 
            \def\PYGZcb{\discretionary{\char`\}}{\Wrappedafterbreak}{\char`\}}}% 
            \def\PYGZca{\discretionary{\char`\^}{\Wrappedafterbreak}{\char`\^}}% 
            \def\PYGZam{\discretionary{\char`\&}{\Wrappedafterbreak}{\char`\&}}% 
            \def\PYGZlt{\discretionary{}{\Wrappedafterbreak\char`\<}{\char`\<}}% 
            \def\PYGZgt{\discretionary{\char`\>}{\Wrappedafterbreak}{\char`\>}}% 
            \def\PYGZsh{\discretionary{}{\Wrappedafterbreak\char`\#}{\char`\#}}% 
            \def\PYGZpc{\discretionary{}{\Wrappedafterbreak\char`\%}{\char`\%}}% 
            \def\PYGZdl{\discretionary{}{\Wrappedafterbreak\char`\$}{\char`\$}}% 
            \def\PYGZhy{\discretionary{\char`\-}{\Wrappedafterbreak}{\char`\-}}% 
            \def\PYGZsq{\discretionary{}{\Wrappedafterbreak\textquotesingle}{\textquotesingle}}% 
            \def\PYGZdq{\discretionary{}{\Wrappedafterbreak\char`\"}{\char`\"}}% 
            \def\PYGZti{\discretionary{\char`\~}{\Wrappedafterbreak}{\char`\~}}% 
        } 
        % Some characters . , ; ? ! / are not pygmentized. 
        % This macro makes them "active" and they will insert potential linebreaks 
        \newcommand*\Wrappedbreaksatpunct {% 
            \lccode`\~`\.\lowercase{\def~}{\discretionary{\hbox{\char`\.}}{\Wrappedafterbreak}{\hbox{\char`\.}}}% 
            \lccode`\~`\,\lowercase{\def~}{\discretionary{\hbox{\char`\,}}{\Wrappedafterbreak}{\hbox{\char`\,}}}% 
            \lccode`\~`\;\lowercase{\def~}{\discretionary{\hbox{\char`\;}}{\Wrappedafterbreak}{\hbox{\char`\;}}}% 
            \lccode`\~`\:\lowercase{\def~}{\discretionary{\hbox{\char`\:}}{\Wrappedafterbreak}{\hbox{\char`\:}}}% 
            \lccode`\~`\?\lowercase{\def~}{\discretionary{\hbox{\char`\?}}{\Wrappedafterbreak}{\hbox{\char`\?}}}% 
            \lccode`\~`\!\lowercase{\def~}{\discretionary{\hbox{\char`\!}}{\Wrappedafterbreak}{\hbox{\char`\!}}}% 
            \lccode`\~`\/\lowercase{\def~}{\discretionary{\hbox{\char`\/}}{\Wrappedafterbreak}{\hbox{\char`\/}}}% 
            \catcode`\.\active
            \catcode`\,\active 
            \catcode`\;\active
            \catcode`\:\active
            \catcode`\?\active
            \catcode`\!\active
            \catcode`\/\active 
            \lccode`\~`\~ 	
        }
    \makeatother

    \let\OriginalVerbatim=\Verbatim
    \makeatletter
    \renewcommand{\Verbatim}[1][1]{%
        %\parskip\z@skip
        \sbox\Wrappedcontinuationbox {\Wrappedcontinuationsymbol}%
        \sbox\Wrappedvisiblespacebox {\FV@SetupFont\Wrappedvisiblespace}%
        \def\FancyVerbFormatLine ##1{\hsize\linewidth
            \vtop{\raggedright\hyphenpenalty\z@\exhyphenpenalty\z@
                \doublehyphendemerits\z@\finalhyphendemerits\z@
                \strut ##1\strut}%
        }%
        % If the linebreak is at a space, the latter will be displayed as visible
        % space at end of first line, and a continuation symbol starts next line.
        % Stretch/shrink are however usually zero for typewriter font.
        \def\FV@Space {%
            \nobreak\hskip\z@ plus\fontdimen3\font minus\fontdimen4\font
            \discretionary{\copy\Wrappedvisiblespacebox}{\Wrappedafterbreak}
            {\kern\fontdimen2\font}%
        }%
        
        % Allow breaks at special characters using \PYG... macros.
        \Wrappedbreaksatspecials
        % Breaks at punctuation characters . , ; ? ! and / need catcode=\active 	
        \OriginalVerbatim[#1,codes*=\Wrappedbreaksatpunct]%
    }
    \makeatother

    % Exact colors from NB
    \definecolor{incolor}{HTML}{303F9F}
    \definecolor{outcolor}{HTML}{D84315}
    \definecolor{cellborder}{HTML}{CFCFCF}
    \definecolor{cellbackground}{HTML}{F7F7F7}
    
    % prompt
    \makeatletter
    \newcommand{\boxspacing}{\kern\kvtcb@left@rule\kern\kvtcb@boxsep}
    \makeatother
    \newcommand{\prompt}[4]{
        \ttfamily\llap{{\color{#2}[#3]:\hspace{3pt}#4}}\vspace{-\baselineskip}
    }
    

    
    % Prevent overflowing lines due to hard-to-break entities
    \sloppy 
    % Setup hyperref package
    \hypersetup{
      breaklinks=true,  % so long urls are correctly broken across lines
      colorlinks=true,
      urlcolor=urlcolor,
      linkcolor=linkcolor,
      citecolor=citecolor,
      }
    % Slightly bigger margins than the latex defaults
    
    \geometry{verbose,tmargin=1in,bmargin=1in,lmargin=1in,rmargin=1in}
    
    

\begin{document}
    
    \maketitle
    
    

    
    \hypertarget{predict-the-median-housing-price-in-california}{%
\section{Predict the Median Housing Price in
California}\label{predict-the-median-housing-price-in-california}}

    \hypertarget{overview}{%
\section{Overview}\label{overview}}

In this chapter we chose the California Housing Prices dataset from the
StatLib repository. This dataset was based on data from the 1990
California census. It is not exactly recent (you could still afford a
nice house in the Bay Area at the time), but it has many qualities for
learning, so we will pretend it is recent data. We also added a
categorical attribute and removed a few features for teaching purposes.

    

    \hypertarget{get-the-data}{%
\section{Get the Data}\label{get-the-data}}

    \hypertarget{libraries}{%
\subsection{Libraries}\label{libraries}}

    \begin{tcolorbox}[breakable, size=fbox, boxrule=1pt, pad at break*=1mm,colback=cellbackground, colframe=cellborder]
\prompt{In}{incolor}{1}{\boxspacing}
\begin{Verbatim}[commandchars=\\\{\}]
\PY{c+c1}{\PYZsh{} Get the important libraries}

\PY{o}{\PYZpc{}}\PY{k}{matplotlib} inline   
\PY{k+kn}{import} \PY{n+nn}{pandas} \PY{k}{as} \PY{n+nn}{pd}
\PY{k+kn}{import} \PY{n+nn}{numpy} \PY{k}{as} \PY{n+nn}{np}
\PY{k+kn}{import} \PY{n+nn}{matplotlib}\PY{n+nn}{.}\PY{n+nn}{pyplot} \PY{k}{as} \PY{n+nn}{plt}
\PY{k+kn}{import} \PY{n+nn}{seaborn} \PY{k}{as} \PY{n+nn}{sns}
\PY{k+kn}{import} \PY{n+nn}{sklearn}
\end{Verbatim}
\end{tcolorbox}

    \hypertarget{read-the-data-and-take-a-quick-look-at-the-data-structure}{%
\subsection{Read the data and Take a Quick Look at the Data
Structure}\label{read-the-data-and-take-a-quick-look-at-the-data-structure}}

    \begin{tcolorbox}[breakable, size=fbox, boxrule=1pt, pad at break*=1mm,colback=cellbackground, colframe=cellborder]
\prompt{In}{incolor}{2}{\boxspacing}
\begin{Verbatim}[commandchars=\\\{\}]
\PY{c+c1}{\PYZsh{} Set the file path}
\PY{n}{path} \PY{o}{=} \PY{l+s+s2}{\PYZdq{}}\PY{l+s+s2}{./data/housing.csv}\PY{l+s+s2}{\PYZdq{}}

\PY{c+c1}{\PYZsh{} Read the data from local}
\PY{n}{raw\PYZus{}data} \PY{o}{=} \PY{n}{pd}\PY{o}{.}\PY{n}{read\PYZus{}csv}\PY{p}{(}\PY{n}{path}\PY{p}{)}

\PY{c+c1}{\PYZsh{} Get the overview of the raw data}
\PY{n}{raw\PYZus{}data}\PY{o}{.}\PY{n}{head}\PY{p}{(}\PY{p}{)}
\end{Verbatim}
\end{tcolorbox}

            \begin{tcolorbox}[breakable, size=fbox, boxrule=.5pt, pad at break*=1mm, opacityfill=0]
\prompt{Out}{outcolor}{2}{\boxspacing}
\begin{Verbatim}[commandchars=\\\{\}]
   longitude  latitude  housing\_median\_age  total\_rooms  total\_bedrooms  \textbackslash{}
0    -122.23     37.88                41.0        880.0           129.0
1    -122.22     37.86                21.0       7099.0          1106.0
2    -122.24     37.85                52.0       1467.0           190.0
3    -122.25     37.85                52.0       1274.0           235.0
4    -122.25     37.85                52.0       1627.0           280.0

   population  households  median\_income  median\_house\_value ocean\_proximity
0       322.0       126.0         8.3252            452600.0        NEAR BAY
1      2401.0      1138.0         8.3014            358500.0        NEAR BAY
2       496.0       177.0         7.2574            352100.0        NEAR BAY
3       558.0       219.0         5.6431            341300.0        NEAR BAY
4       565.0       259.0         3.8462            342200.0        NEAR BAY
\end{Verbatim}
\end{tcolorbox}
        
    \hypertarget{preview-some-important-info}{%
\subsection{Preview some important
info}\label{preview-some-important-info}}

    \begin{tcolorbox}[breakable, size=fbox, boxrule=1pt, pad at break*=1mm,colback=cellbackground, colframe=cellborder]
\prompt{In}{incolor}{3}{\boxspacing}
\begin{Verbatim}[commandchars=\\\{\}]
\PY{n}{raw\PYZus{}data}\PY{o}{.}\PY{n}{info}\PY{p}{(}\PY{p}{)}
\end{Verbatim}
\end{tcolorbox}

    \begin{Verbatim}[commandchars=\\\{\}]
<class 'pandas.core.frame.DataFrame'>
RangeIndex: 20640 entries, 0 to 20639
Data columns (total 10 columns):
 \#   Column              Non-Null Count  Dtype
---  ------              --------------  -----
 0   longitude           20640 non-null  float64
 1   latitude            20640 non-null  float64
 2   housing\_median\_age  20640 non-null  float64
 3   total\_rooms         20640 non-null  float64
 4   total\_bedrooms      20433 non-null  float64
 5   population          20640 non-null  float64
 6   households          20640 non-null  float64
 7   median\_income       20640 non-null  float64
 8   median\_house\_value  20640 non-null  float64
 9   ocean\_proximity     20640 non-null  object
dtypes: float64(9), object(1)
memory usage: 1.6+ MB
    \end{Verbatim}

    Analize how many districts belong to each category

    \begin{tcolorbox}[breakable, size=fbox, boxrule=1pt, pad at break*=1mm,colback=cellbackground, colframe=cellborder]
\prompt{In}{incolor}{4}{\boxspacing}
\begin{Verbatim}[commandchars=\\\{\}]
\PY{c+c1}{\PYZsh{} Get the values of districts of each ocean\PYZus{}proximity category}
\PY{n}{raw\PYZus{}data}\PY{p}{[}\PY{l+s+s2}{\PYZdq{}}\PY{l+s+s2}{ocean\PYZus{}proximity}\PY{l+s+s2}{\PYZdq{}}\PY{p}{]}\PY{o}{.}\PY{n}{value\PYZus{}counts}\PY{p}{(}\PY{p}{)}
\end{Verbatim}
\end{tcolorbox}

            \begin{tcolorbox}[breakable, size=fbox, boxrule=.5pt, pad at break*=1mm, opacityfill=0]
\prompt{Out}{outcolor}{4}{\boxspacing}
\begin{Verbatim}[commandchars=\\\{\}]
<1H OCEAN     9136
INLAND        6551
NEAR OCEAN    2658
NEAR BAY      2290
ISLAND           5
Name: ocean\_proximity, dtype: int64
\end{Verbatim}
\end{tcolorbox}
        
    \begin{tcolorbox}[breakable, size=fbox, boxrule=1pt, pad at break*=1mm,colback=cellbackground, colframe=cellborder]
\prompt{In}{incolor}{5}{\boxspacing}
\begin{Verbatim}[commandchars=\\\{\}]
\PY{c+c1}{\PYZsh{} Get the summary of the numerical attributes}
\PY{n}{raw\PYZus{}data}\PY{o}{.}\PY{n}{describe}\PY{p}{(}\PY{p}{)}
\end{Verbatim}
\end{tcolorbox}

            \begin{tcolorbox}[breakable, size=fbox, boxrule=.5pt, pad at break*=1mm, opacityfill=0]
\prompt{Out}{outcolor}{5}{\boxspacing}
\begin{Verbatim}[commandchars=\\\{\}]
          longitude      latitude  housing\_median\_age   total\_rooms  \textbackslash{}
count  20640.000000  20640.000000        20640.000000  20640.000000
mean    -119.569704     35.631861           28.639486   2635.763081
std        2.003532      2.135952           12.585558   2181.615252
min     -124.350000     32.540000            1.000000      2.000000
25\%     -121.800000     33.930000           18.000000   1447.750000
50\%     -118.490000     34.260000           29.000000   2127.000000
75\%     -118.010000     37.710000           37.000000   3148.000000
max     -114.310000     41.950000           52.000000  39320.000000

       total\_bedrooms    population    households  median\_income  \textbackslash{}
count    20433.000000  20640.000000  20640.000000   20640.000000
mean       537.870553   1425.476744    499.539680       3.870671
std        421.385070   1132.462122    382.329753       1.899822
min          1.000000      3.000000      1.000000       0.499900
25\%        296.000000    787.000000    280.000000       2.563400
50\%        435.000000   1166.000000    409.000000       3.534800
75\%        647.000000   1725.000000    605.000000       4.743250
max       6445.000000  35682.000000   6082.000000      15.000100

       median\_house\_value
count        20640.000000
mean        206855.816909
std         115395.615874
min          14999.000000
25\%         119600.000000
50\%         179700.000000
75\%         264725.000000
max         500001.000000
\end{Verbatim}
\end{tcolorbox}
        
    Plot the attributes in a Histogram for get another point of view.

    \begin{tcolorbox}[breakable, size=fbox, boxrule=1pt, pad at break*=1mm,colback=cellbackground, colframe=cellborder]
\prompt{In}{incolor}{6}{\boxspacing}
\begin{Verbatim}[commandchars=\\\{\}]
\PY{c+c1}{\PYZsh{} Plot a histogram}
\PY{n}{raw\PYZus{}data}\PY{o}{.}\PY{n}{hist}\PY{p}{(}\PY{n}{bins}\PY{o}{=}\PY{l+m+mi}{50}\PY{p}{,} \PY{n}{figsize}\PY{o}{=}\PY{p}{(}\PY{l+m+mi}{20}\PY{p}{,}\PY{l+m+mi}{15}\PY{p}{)}\PY{p}{)}
\PY{n}{plt}\PY{o}{.}\PY{n}{show}\PY{p}{(}\PY{p}{)}
\end{Verbatim}
\end{tcolorbox}

    \begin{center}
    \adjustimage{max size={0.9\linewidth}{0.9\paperheight}}{output_14_0.png}
    \end{center}
    { \hspace*{\fill} \\}
    
    \hypertarget{data-preparation}{%
\section{Data preparation}\label{data-preparation}}

    \hypertarget{split-the-data-into-train-and-test-dataset}{%
\subsection{Split the data into train and test
dataset}\label{split-the-data-into-train-and-test-dataset}}

    \begin{tcolorbox}[breakable, size=fbox, boxrule=1pt, pad at break*=1mm,colback=cellbackground, colframe=cellborder]
\prompt{In}{incolor}{7}{\boxspacing}
\begin{Verbatim}[commandchars=\\\{\}]
\PY{n}{data} \PY{o}{=} \PY{n}{raw\PYZus{}data}
\end{Verbatim}
\end{tcolorbox}

    Split the data randomly

    \begin{tcolorbox}[breakable, size=fbox, boxrule=1pt, pad at break*=1mm,colback=cellbackground, colframe=cellborder]
\prompt{In}{incolor}{8}{\boxspacing}
\begin{Verbatim}[commandchars=\\\{\}]
\PY{k+kn}{from} \PY{n+nn}{sklearn}\PY{n+nn}{.}\PY{n+nn}{model\PYZus{}selection} \PY{k+kn}{import} \PY{n}{train\PYZus{}test\PYZus{}split}

\PY{c+c1}{\PYZsh{} random\PYZus{}state parameter makes replicable}
\PY{n}{train\PYZus{}set}\PY{p}{,} \PY{n}{test\PYZus{}set} \PY{o}{=} \PY{n}{train\PYZus{}test\PYZus{}split}\PY{p}{(}\PY{n}{data}\PY{p}{,} \PY{n}{test\PYZus{}size}\PY{o}{=}\PY{l+m+mf}{0.2}\PY{p}{,} \PY{n}{random\PYZus{}state}\PY{o}{=}\PY{l+m+mi}{42}\PY{p}{)}
\end{Verbatim}
\end{tcolorbox}

    Create a categorical attribute for income median values

    \begin{tcolorbox}[breakable, size=fbox, boxrule=1pt, pad at break*=1mm,colback=cellbackground, colframe=cellborder]
\prompt{In}{incolor}{9}{\boxspacing}
\begin{Verbatim}[commandchars=\\\{\}]
\PY{n}{data}\PY{p}{[}\PY{l+s+s2}{\PYZdq{}}\PY{l+s+s2}{income\PYZus{}cat}\PY{l+s+s2}{\PYZdq{}}\PY{p}{]} \PY{o}{=} \PY{n}{pd}\PY{o}{.}\PY{n}{cut}\PY{p}{(}\PY{n}{data}\PY{p}{[}\PY{l+s+s2}{\PYZdq{}}\PY{l+s+s2}{median\PYZus{}income}\PY{l+s+s2}{\PYZdq{}}\PY{p}{]}\PY{p}{,}
                               \PY{n}{bins}\PY{o}{=}\PY{p}{[}\PY{l+m+mf}{0.}\PY{p}{,} \PY{l+m+mf}{1.5}\PY{p}{,} \PY{l+m+mf}{3.0}\PY{p}{,} \PY{l+m+mf}{4.5}\PY{p}{,} \PY{l+m+mf}{6.}\PY{p}{,} \PY{n}{np}\PY{o}{.}\PY{n}{inf}\PY{p}{]}\PY{p}{,}
                               \PY{n}{labels}\PY{o}{=}\PY{p}{[}\PY{l+m+mi}{1}\PY{p}{,} \PY{l+m+mi}{2}\PY{p}{,} \PY{l+m+mi}{3}\PY{p}{,} \PY{l+m+mi}{4}\PY{p}{,} \PY{l+m+mi}{5}\PY{p}{]}\PY{p}{)}
\end{Verbatim}
\end{tcolorbox}

    Split the data using stratified sampling

    \begin{tcolorbox}[breakable, size=fbox, boxrule=1pt, pad at break*=1mm,colback=cellbackground, colframe=cellborder]
\prompt{In}{incolor}{10}{\boxspacing}
\begin{Verbatim}[commandchars=\\\{\}]
\PY{k+kn}{from} \PY{n+nn}{sklearn}\PY{n+nn}{.}\PY{n+nn}{model\PYZus{}selection} \PY{k+kn}{import} \PY{n}{StratifiedShuffleSplit}

\PY{c+c1}{\PYZsh{} Split the data}
\PY{n}{split} \PY{o}{=} \PY{n}{StratifiedShuffleSplit}\PY{p}{(}\PY{n}{n\PYZus{}splits}\PY{o}{=}\PY{l+m+mi}{1}\PY{p}{,} \PY{n}{test\PYZus{}size}\PY{o}{=}\PY{l+m+mf}{0.2}\PY{p}{,} \PY{n}{random\PYZus{}state}\PY{o}{=}\PY{l+m+mi}{42}\PY{p}{)}
\PY{k}{for} \PY{n}{train\PYZus{}index}\PY{p}{,} \PY{n}{test\PYZus{}index} \PY{o+ow}{in} \PY{n}{split}\PY{o}{.}\PY{n}{split}\PY{p}{(}\PY{n}{data}\PY{p}{,} \PY{n}{data}\PY{p}{[}\PY{l+s+s2}{\PYZdq{}}\PY{l+s+s2}{income\PYZus{}cat}\PY{l+s+s2}{\PYZdq{}}\PY{p}{]}\PY{p}{)}\PY{p}{:}
    \PY{n}{strat\PYZus{}train\PYZus{}set} \PY{o}{=} \PY{n}{data}\PY{o}{.}\PY{n}{loc}\PY{p}{[}\PY{n}{train\PYZus{}index}\PY{p}{]}
    \PY{n}{strat\PYZus{}test\PYZus{}set} \PY{o}{=} \PY{n}{data}\PY{o}{.}\PY{n}{loc}\PY{p}{[}\PY{n}{test\PYZus{}index}\PY{p}{]}
\end{Verbatim}
\end{tcolorbox}

    Drop the categorical attirbute created before, and return to the
original state

    \begin{tcolorbox}[breakable, size=fbox, boxrule=1pt, pad at break*=1mm,colback=cellbackground, colframe=cellborder]
\prompt{In}{incolor}{11}{\boxspacing}
\begin{Verbatim}[commandchars=\\\{\}]
\PY{k}{for} \PY{n}{set\PYZus{}} \PY{o+ow}{in} \PY{p}{(}\PY{n}{strat\PYZus{}train\PYZus{}set}\PY{p}{,} \PY{n}{strat\PYZus{}test\PYZus{}set}\PY{p}{)}\PY{p}{:}
    \PY{n}{set\PYZus{}}\PY{o}{.}\PY{n}{drop}\PY{p}{(}\PY{l+s+s2}{\PYZdq{}}\PY{l+s+s2}{income\PYZus{}cat}\PY{l+s+s2}{\PYZdq{}}\PY{p}{,} \PY{n}{axis}\PY{o}{=}\PY{l+m+mi}{1}\PY{p}{,} \PY{n}{inplace}\PY{o}{=}\PY{k+kc}{True}\PY{p}{)}
\end{Verbatim}
\end{tcolorbox}

    \hypertarget{discover-and-visualize-the-data-to-gain-insights}{%
\section{Discover and Visualize the data to gain
Insights}\label{discover-and-visualize-the-data-to-gain-insights}}

    \begin{tcolorbox}[breakable, size=fbox, boxrule=1pt, pad at break*=1mm,colback=cellbackground, colframe=cellborder]
\prompt{In}{incolor}{12}{\boxspacing}
\begin{Verbatim}[commandchars=\\\{\}]
\PY{c+c1}{\PYZsh{} Make a copy of the training data}
\PY{n}{housing} \PY{o}{=} \PY{n}{strat\PYZus{}train\PYZus{}set}\PY{o}{.}\PY{n}{copy}\PY{p}{(}\PY{p}{)}
\end{Verbatim}
\end{tcolorbox}

    \hypertarget{view-geographical-data}{%
\subsection{View geographical data}\label{view-geographical-data}}

Since there is a geographical information, it is a good idea to create a
scatterplot of all districts to visualize the data. The value of alpha =
0.1 makes it much easier to visualize the places where there is a high
density of data points.

    \begin{tcolorbox}[breakable, size=fbox, boxrule=1pt, pad at break*=1mm,colback=cellbackground, colframe=cellborder]
\prompt{In}{incolor}{13}{\boxspacing}
\begin{Verbatim}[commandchars=\\\{\}]
\PY{c+c1}{\PYZsh{} Plotting geographical data}
\PY{n}{housing}\PY{o}{.}\PY{n}{plot}\PY{p}{(}\PY{n}{kind}\PY{o}{=}\PY{l+s+s2}{\PYZdq{}}\PY{l+s+s2}{scatter}\PY{l+s+s2}{\PYZdq{}}\PY{p}{,} \PY{n}{x}\PY{o}{=}\PY{l+s+s2}{\PYZdq{}}\PY{l+s+s2}{longitude}\PY{l+s+s2}{\PYZdq{}}\PY{p}{,} \PY{n}{y}\PY{o}{=}\PY{l+s+s2}{\PYZdq{}}\PY{l+s+s2}{latitude}\PY{l+s+s2}{\PYZdq{}}\PY{p}{,} \PY{n}{alpha}\PY{o}{=}\PY{l+m+mf}{0.1}\PY{p}{)}
\end{Verbatim}
\end{tcolorbox}

            \begin{tcolorbox}[breakable, size=fbox, boxrule=.5pt, pad at break*=1mm, opacityfill=0]
\prompt{Out}{outcolor}{13}{\boxspacing}
\begin{Verbatim}[commandchars=\\\{\}]
<matplotlib.axes.\_subplots.AxesSubplot at 0x20816f17880>
\end{Verbatim}
\end{tcolorbox}
        
    \begin{center}
    \adjustimage{max size={0.9\linewidth}{0.9\paperheight}}{output_29_1.png}
    \end{center}
    { \hspace*{\fill} \\}
    
    Now let´s look at the prices information, in this ocation the we will
play with some options for obtain a better representation of the data.
The radius of each circle (Option s) represents the distric's
population, the color represents the price (Option c). At the same time
a predefined color map it is used (Option cmap) caled jet.

    \begin{tcolorbox}[breakable, size=fbox, boxrule=1pt, pad at break*=1mm,colback=cellbackground, colframe=cellborder]
\prompt{In}{incolor}{14}{\boxspacing}
\begin{Verbatim}[commandchars=\\\{\}]
\PY{n}{housing}\PY{o}{.}\PY{n}{plot}\PY{p}{(}\PY{n}{kind}\PY{o}{=}\PY{l+s+s2}{\PYZdq{}}\PY{l+s+s2}{scatter}\PY{l+s+s2}{\PYZdq{}}\PY{p}{,} \PY{n}{x}\PY{o}{=}\PY{l+s+s2}{\PYZdq{}}\PY{l+s+s2}{longitude}\PY{l+s+s2}{\PYZdq{}}\PY{p}{,} \PY{n}{y}\PY{o}{=}\PY{l+s+s2}{\PYZdq{}}\PY{l+s+s2}{latitude}\PY{l+s+s2}{\PYZdq{}}\PY{p}{,} \PY{n}{alpha}\PY{o}{=}\PY{l+m+mf}{0.4}\PY{p}{,}
    \PY{n}{s}\PY{o}{=}\PY{n}{housing}\PY{p}{[}\PY{l+s+s2}{\PYZdq{}}\PY{l+s+s2}{population}\PY{l+s+s2}{\PYZdq{}}\PY{p}{]}\PY{o}{/}\PY{l+m+mi}{100}\PY{p}{,} \PY{n}{label}\PY{o}{=}\PY{l+s+s2}{\PYZdq{}}\PY{l+s+s2}{population}\PY{l+s+s2}{\PYZdq{}}\PY{p}{,} \PY{n}{figsize}\PY{o}{=}\PY{p}{(}\PY{l+m+mi}{10}\PY{p}{,}\PY{l+m+mi}{7}\PY{p}{)}\PY{p}{,}
    \PY{n}{c}\PY{o}{=}\PY{l+s+s2}{\PYZdq{}}\PY{l+s+s2}{median\PYZus{}house\PYZus{}value}\PY{l+s+s2}{\PYZdq{}}\PY{p}{,} \PY{n}{cmap}\PY{o}{=}\PY{n}{plt}\PY{o}{.}\PY{n}{get\PYZus{}cmap}\PY{p}{(}\PY{l+s+s2}{\PYZdq{}}\PY{l+s+s2}{jet}\PY{l+s+s2}{\PYZdq{}}\PY{p}{)}\PY{p}{,} \PY{n}{colorbar}\PY{o}{=}\PY{k+kc}{True}\PY{p}{,}
\PY{p}{)}
\PY{n}{plt}\PY{o}{.}\PY{n}{legend}\PY{p}{(}\PY{p}{)}
\end{Verbatim}
\end{tcolorbox}

            \begin{tcolorbox}[breakable, size=fbox, boxrule=.5pt, pad at break*=1mm, opacityfill=0]
\prompt{Out}{outcolor}{14}{\boxspacing}
\begin{Verbatim}[commandchars=\\\{\}]
<matplotlib.legend.Legend at 0x20816ffbb20>
\end{Verbatim}
\end{tcolorbox}
        
    \begin{center}
    \adjustimage{max size={0.9\linewidth}{0.9\paperheight}}{output_31_1.png}
    \end{center}
    { \hspace*{\fill} \\}
    
    This image tell us that the housing pricing are very much related to the
location (e.g, close to the ocean) and to the population density. It
will be probably be useful to use a clustering algorithm to detect the
main clusters and add new features that measure the proximity to the
cluster centers.

    \hypertarget{looking-for-correlations}{%
\subsection{Looking for Correlations}\label{looking-for-correlations}}

    In this section it will be calculated the correlations, in particular
the ``standard correlation coefficient (Pearson's r)'' this values
measures how an attribute affects each other.

    \begin{tcolorbox}[breakable, size=fbox, boxrule=1pt, pad at break*=1mm,colback=cellbackground, colframe=cellborder]
\prompt{In}{incolor}{15}{\boxspacing}
\begin{Verbatim}[commandchars=\\\{\}]
\PY{c+c1}{\PYZsh{} Get the correlation values between the attributes}
\PY{n}{corr\PYZus{}matrix} \PY{o}{=} \PY{n}{housing}\PY{o}{.}\PY{n}{corr}\PY{p}{(}\PY{p}{)}
\end{Verbatim}
\end{tcolorbox}

    Now let's look at how much each attribute correlates with the median
house value:

    \begin{tcolorbox}[breakable, size=fbox, boxrule=1pt, pad at break*=1mm,colback=cellbackground, colframe=cellborder]
\prompt{In}{incolor}{16}{\boxspacing}
\begin{Verbatim}[commandchars=\\\{\}]
\PY{c+c1}{\PYZsh{} Get the correlations vector related to the target variable}
\PY{n}{corr\PYZus{}matrix}\PY{p}{[}\PY{l+s+s2}{\PYZdq{}}\PY{l+s+s2}{median\PYZus{}house\PYZus{}value}\PY{l+s+s2}{\PYZdq{}}\PY{p}{]}\PY{o}{.}\PY{n}{sort\PYZus{}values}\PY{p}{(}\PY{n}{ascending}\PY{o}{=}\PY{k+kc}{False}\PY{p}{)}
\end{Verbatim}
\end{tcolorbox}

            \begin{tcolorbox}[breakable, size=fbox, boxrule=.5pt, pad at break*=1mm, opacityfill=0]
\prompt{Out}{outcolor}{16}{\boxspacing}
\begin{Verbatim}[commandchars=\\\{\}]
median\_house\_value    1.000000
median\_income         0.687160
total\_rooms           0.135097
housing\_median\_age    0.114110
households            0.064506
total\_bedrooms        0.047689
population           -0.026920
longitude            -0.047432
latitude             -0.142724
Name: median\_house\_value, dtype: float64
\end{Verbatim}
\end{tcolorbox}
        
    It is clear that the most influential variable in the houses values is
the value of the median\_income variable.

    Another way to check for correlation between attributes is to use
Pandas' scatter\_matrix function, which plots every numerical attribute
against every other numerical attribute (target included). In this plot
the main diagonal shows a histogram of each attribute, due to the
corresponding correlation against itself it is one and this factor would
not be very useful.

    \begin{tcolorbox}[breakable, size=fbox, boxrule=1pt, pad at break*=1mm,colback=cellbackground, colframe=cellborder]
\prompt{In}{incolor}{17}{\boxspacing}
\begin{Verbatim}[commandchars=\\\{\}]
\PY{k+kn}{from} \PY{n+nn}{pandas}\PY{n+nn}{.}\PY{n+nn}{plotting} \PY{k+kn}{import} \PY{n}{scatter\PYZus{}matrix}

\PY{n}{attributes} \PY{o}{=} \PY{p}{[}\PY{l+s+s2}{\PYZdq{}}\PY{l+s+s2}{median\PYZus{}house\PYZus{}value}\PY{l+s+s2}{\PYZdq{}}\PY{p}{,} \PY{l+s+s2}{\PYZdq{}}\PY{l+s+s2}{median\PYZus{}income}\PY{l+s+s2}{\PYZdq{}}\PY{p}{,} \PY{l+s+s2}{\PYZdq{}}\PY{l+s+s2}{total\PYZus{}rooms}\PY{l+s+s2}{\PYZdq{}}\PY{p}{,}
              \PY{l+s+s2}{\PYZdq{}}\PY{l+s+s2}{housing\PYZus{}median\PYZus{}age}\PY{l+s+s2}{\PYZdq{}}\PY{p}{]}
\PY{n}{scatter\PYZus{}matrix}\PY{p}{(}\PY{n}{housing}\PY{p}{[}\PY{n}{attributes}\PY{p}{]}\PY{p}{,} \PY{n}{figsize}\PY{o}{=}\PY{p}{(}\PY{l+m+mi}{12}\PY{p}{,} \PY{l+m+mi}{8}\PY{p}{)}\PY{p}{)}
\end{Verbatim}
\end{tcolorbox}

            \begin{tcolorbox}[breakable, size=fbox, boxrule=.5pt, pad at break*=1mm, opacityfill=0]
\prompt{Out}{outcolor}{17}{\boxspacing}
\begin{Verbatim}[commandchars=\\\{\}]
array([[<matplotlib.axes.\_subplots.AxesSubplot object at 0x000002081718B850>,
        <matplotlib.axes.\_subplots.AxesSubplot object at 0x0000020816EFB670>,
        <matplotlib.axes.\_subplots.AxesSubplot object at 0x0000020816ECE430>,
        <matplotlib.axes.\_subplots.AxesSubplot object at 0x000002081716D6D0>],
       [<matplotlib.axes.\_subplots.AxesSubplot object at 0x00000208170425B0>,
        <matplotlib.axes.\_subplots.AxesSubplot object at 0x0000020817AA7970>,
        <matplotlib.axes.\_subplots.AxesSubplot object at 0x0000020817AA7A60>,
        <matplotlib.axes.\_subplots.AxesSubplot object at 0x00000208170A8F10>],
       [<matplotlib.axes.\_subplots.AxesSubplot object at 0x0000020816FC2760>,
        <matplotlib.axes.\_subplots.AxesSubplot object at 0x0000020816F73BB0>,
        <matplotlib.axes.\_subplots.AxesSubplot object at 0x0000020816EC90A0>,
        <matplotlib.axes.\_subplots.AxesSubplot object at 0x000002081707A490>],
       [<matplotlib.axes.\_subplots.AxesSubplot object at 0x00000208171358E0>,
        <matplotlib.axes.\_subplots.AxesSubplot object at 0x0000020816F30D30>,
        <matplotlib.axes.\_subplots.AxesSubplot object at 0x00000208171AD1C0>,
        <matplotlib.axes.\_subplots.AxesSubplot object at 0x00000208171EA610>]],
      dtype=object)
\end{Verbatim}
\end{tcolorbox}
        
    \begin{center}
    \adjustimage{max size={0.9\linewidth}{0.9\paperheight}}{output_40_1.png}
    \end{center}
    { \hspace*{\fill} \\}
    
    \hypertarget{attribute-combinations}{%
\section{Attribute Combinations}\label{attribute-combinations}}

    In this section we will try out various attribute combinations, using
the actual attribute combined with other and together make a new one.
For example can be intresting to compute the value os the number of
rooms in each huoseholds, or the total number of bedrooms per room. In
the next chunk, we will compute the new attributes.

    \begin{tcolorbox}[breakable, size=fbox, boxrule=1pt, pad at break*=1mm,colback=cellbackground, colframe=cellborder]
\prompt{In}{incolor}{18}{\boxspacing}
\begin{Verbatim}[commandchars=\\\{\}]
\PY{n}{housing}\PY{p}{[}\PY{l+s+s2}{\PYZdq{}}\PY{l+s+s2}{rooms\PYZus{}per\PYZus{}household}\PY{l+s+s2}{\PYZdq{}}\PY{p}{]} \PY{o}{=} \PY{n}{housing}\PY{p}{[}\PY{l+s+s2}{\PYZdq{}}\PY{l+s+s2}{total\PYZus{}rooms}\PY{l+s+s2}{\PYZdq{}}\PY{p}{]}\PY{o}{/}\PY{n}{housing}\PY{p}{[}\PY{l+s+s2}{\PYZdq{}}\PY{l+s+s2}{households}\PY{l+s+s2}{\PYZdq{}}\PY{p}{]}
\PY{n}{housing}\PY{p}{[}\PY{l+s+s2}{\PYZdq{}}\PY{l+s+s2}{bedrooms\PYZus{}per\PYZus{}room}\PY{l+s+s2}{\PYZdq{}}\PY{p}{]} \PY{o}{=} \PY{n}{housing}\PY{p}{[}\PY{l+s+s2}{\PYZdq{}}\PY{l+s+s2}{total\PYZus{}bedrooms}\PY{l+s+s2}{\PYZdq{}}\PY{p}{]}\PY{o}{/}\PY{n}{housing}\PY{p}{[}\PY{l+s+s2}{\PYZdq{}}\PY{l+s+s2}{total\PYZus{}rooms}\PY{l+s+s2}{\PYZdq{}}\PY{p}{]}
\PY{n}{housing}\PY{p}{[}\PY{l+s+s2}{\PYZdq{}}\PY{l+s+s2}{population\PYZus{}per\PYZus{}household}\PY{l+s+s2}{\PYZdq{}}\PY{p}{]}\PY{o}{=}\PY{n}{housing}\PY{p}{[}\PY{l+s+s2}{\PYZdq{}}\PY{l+s+s2}{population}\PY{l+s+s2}{\PYZdq{}}\PY{p}{]}\PY{o}{/}\PY{n}{housing}\PY{p}{[}\PY{l+s+s2}{\PYZdq{}}\PY{l+s+s2}{households}\PY{l+s+s2}{\PYZdq{}}\PY{p}{]}

\PY{n}{corr\PYZus{}matrix} \PY{o}{=} \PY{n}{housing}\PY{o}{.}\PY{n}{corr}\PY{p}{(}\PY{p}{)}
\PY{n}{corr\PYZus{}matrix}\PY{p}{[}\PY{l+s+s2}{\PYZdq{}}\PY{l+s+s2}{median\PYZus{}house\PYZus{}value}\PY{l+s+s2}{\PYZdq{}}\PY{p}{]}\PY{o}{.}\PY{n}{sort\PYZus{}values}\PY{p}{(}\PY{n}{ascending}\PY{o}{=}\PY{k+kc}{False}\PY{p}{)}
\end{Verbatim}
\end{tcolorbox}

            \begin{tcolorbox}[breakable, size=fbox, boxrule=.5pt, pad at break*=1mm, opacityfill=0]
\prompt{Out}{outcolor}{18}{\boxspacing}
\begin{Verbatim}[commandchars=\\\{\}]
median\_house\_value          1.000000
median\_income               0.687160
rooms\_per\_household         0.146285
total\_rooms                 0.135097
housing\_median\_age          0.114110
households                  0.064506
total\_bedrooms              0.047689
population\_per\_household   -0.021985
population                 -0.026920
longitude                  -0.047432
latitude                   -0.142724
bedrooms\_per\_room          -0.259984
Name: median\_house\_value, dtype: float64
\end{Verbatim}
\end{tcolorbox}
        
    As you can see, the new value of bedrooms\_per\_room it it much more
correlated with th median value than the total number or rooms/bedrooms,
apparently, houses with a lower bedroom/room ratio tend to be more
expensive.

    \hypertarget{data-preparation}{%
\section{Data Preparation}\label{data-preparation}}

    The first step in this section will be come back toa a clean training
dataset, and let's separate the predictors and the labels since we don't
necessary want to apply same transfomrations to the predictors and the
target values.

    \begin{tcolorbox}[breakable, size=fbox, boxrule=1pt, pad at break*=1mm,colback=cellbackground, colframe=cellborder]
\prompt{In}{incolor}{19}{\boxspacing}
\begin{Verbatim}[commandchars=\\\{\}]
\PY{n}{housing} \PY{o}{=} \PY{n}{strat\PYZus{}train\PYZus{}set}\PY{o}{.}\PY{n}{drop}\PY{p}{(}\PY{l+s+s2}{\PYZdq{}}\PY{l+s+s2}{median\PYZus{}house\PYZus{}value}\PY{l+s+s2}{\PYZdq{}}\PY{p}{,} \PY{n}{axis}\PY{o}{=}\PY{l+m+mi}{1}\PY{p}{)}
\PY{n}{housing\PYZus{}labels} \PY{o}{=} \PY{n}{strat\PYZus{}train\PYZus{}set}\PY{p}{[}\PY{l+s+s2}{\PYZdq{}}\PY{l+s+s2}{median\PYZus{}house\PYZus{}value}\PY{l+s+s2}{\PYZdq{}}\PY{p}{]}\PY{o}{.}\PY{n}{copy}\PY{p}{(}\PY{p}{)}
\end{Verbatim}
\end{tcolorbox}

    \textbf{Tip}: Instead of just doing manually, write functions, it will
be better for makes reproducible and scalable. Also, you can test all
your transformation and mixe it eachother and compare which works better
for your code. Finally you can use these functions in a live system to
transform the new data before fedding the algorirthm.

    \hypertarget{data-cleaning}{%
\subsection{Data Cleaning}\label{data-cleaning}}

    Firstly, we will trate the missing values, as we noticed earlier the
feature \emph{total\_bedrooms} has some missing values, so let's create
some functions to take care of them.

Scikit-Learn provides a handy class to take care of missing values:
SimpleImputer. Here is how to use it. First, you need to create a
SimpleImputer instance, specifying that you want to replace each
attribute's missing values with the median of that attribute:

    \begin{tcolorbox}[breakable, size=fbox, boxrule=1pt, pad at break*=1mm,colback=cellbackground, colframe=cellborder]
\prompt{In}{incolor}{20}{\boxspacing}
\begin{Verbatim}[commandchars=\\\{\}]
\PY{k+kn}{from} \PY{n+nn}{sklearn}\PY{n+nn}{.}\PY{n+nn}{impute} \PY{k+kn}{import} \PY{n}{SimpleImputer}

\PY{c+c1}{\PYZsh{} Create the imputer}
\PY{n}{imputer} \PY{o}{=} \PY{n}{SimpleImputer}\PY{p}{(}\PY{n}{strategy}\PY{o}{=}\PY{l+s+s2}{\PYZdq{}}\PY{l+s+s2}{median}\PY{l+s+s2}{\PYZdq{}}\PY{p}{)}
\end{Verbatim}
\end{tcolorbox}

    Since the median can only be computed on numerical attributes, so we
need to create a copy of our data without categorical features.

    \begin{tcolorbox}[breakable, size=fbox, boxrule=1pt, pad at break*=1mm,colback=cellbackground, colframe=cellborder]
\prompt{In}{incolor}{21}{\boxspacing}
\begin{Verbatim}[commandchars=\\\{\}]
\PY{c+c1}{\PYZsh{} Drop the categorical variable}
\PY{n}{housing\PYZus{}num} \PY{o}{=} \PY{n}{housing}\PY{o}{.}\PY{n}{drop}\PY{p}{(}\PY{l+s+s2}{\PYZdq{}}\PY{l+s+s2}{ocean\PYZus{}proximity}\PY{l+s+s2}{\PYZdq{}}\PY{p}{,} \PY{n}{axis}\PY{o}{=}\PY{l+m+mi}{1}\PY{p}{)}
\end{Verbatim}
\end{tcolorbox}

    Now we can fit the imputer with the numerical dataframe.

    \begin{tcolorbox}[breakable, size=fbox, boxrule=1pt, pad at break*=1mm,colback=cellbackground, colframe=cellborder]
\prompt{In}{incolor}{22}{\boxspacing}
\begin{Verbatim}[commandchars=\\\{\}]
\PY{n}{imputer}\PY{o}{.}\PY{n}{fit}\PY{p}{(}\PY{n}{housing\PYZus{}num}\PY{p}{)}
\end{Verbatim}
\end{tcolorbox}

            \begin{tcolorbox}[breakable, size=fbox, boxrule=.5pt, pad at break*=1mm, opacityfill=0]
\prompt{Out}{outcolor}{22}{\boxspacing}
\begin{Verbatim}[commandchars=\\\{\}]
SimpleImputer(strategy='median')
\end{Verbatim}
\end{tcolorbox}
        
    Now it is time to transform the training dataset by replacing missing
values by the median value.

    \begin{tcolorbox}[breakable, size=fbox, boxrule=1pt, pad at break*=1mm,colback=cellbackground, colframe=cellborder]
\prompt{In}{incolor}{23}{\boxspacing}
\begin{Verbatim}[commandchars=\\\{\}]
\PY{n}{X} \PY{o}{=} \PY{n}{imputer}\PY{o}{.}\PY{n}{transform}\PY{p}{(}\PY{n}{housing\PYZus{}num}\PY{p}{)}
\end{Verbatim}
\end{tcolorbox}

    The result is a plain NumPy array, if you wanto to put it back into a
Pandas dataframe it is simple.

    \begin{tcolorbox}[breakable, size=fbox, boxrule=1pt, pad at break*=1mm,colback=cellbackground, colframe=cellborder]
\prompt{In}{incolor}{24}{\boxspacing}
\begin{Verbatim}[commandchars=\\\{\}]
\PY{c+c1}{\PYZsh{} Create a Pandas dataframe from NumPy array.}
\PY{n}{housing\PYZus{}tr} \PY{o}{=} \PY{n}{pd}\PY{o}{.}\PY{n}{DataFrame}\PY{p}{(}\PY{n}{X}\PY{p}{,} \PY{n}{columns}\PY{o}{=}\PY{n}{housing\PYZus{}num}\PY{o}{.}\PY{n}{columns}\PY{p}{)}
\PY{n}{housing\PYZus{}tr}\PY{o}{.}\PY{n}{info}\PY{p}{(}\PY{p}{)}
\end{Verbatim}
\end{tcolorbox}

    \begin{Verbatim}[commandchars=\\\{\}]
<class 'pandas.core.frame.DataFrame'>
RangeIndex: 16512 entries, 0 to 16511
Data columns (total 8 columns):
 \#   Column              Non-Null Count  Dtype
---  ------              --------------  -----
 0   longitude           16512 non-null  float64
 1   latitude            16512 non-null  float64
 2   housing\_median\_age  16512 non-null  float64
 3   total\_rooms         16512 non-null  float64
 4   total\_bedrooms      16512 non-null  float64
 5   population          16512 non-null  float64
 6   households          16512 non-null  float64
 7   median\_income       16512 non-null  float64
dtypes: float64(8)
memory usage: 1.0 MB
    \end{Verbatim}

    \hypertarget{text-and-categorical-attributes}{%
\subsection{Text and Categorical
Attributes}\label{text-and-categorical-attributes}}

    Most Machine Learning algorithms prefer to work with numbers anyway, so
let's convert these categories from text to numbers. For this, we can
use Scikit-Learn's OrdinalEncoder class

    \begin{tcolorbox}[breakable, size=fbox, boxrule=1pt, pad at break*=1mm,colback=cellbackground, colframe=cellborder]
\prompt{In}{incolor}{25}{\boxspacing}
\begin{Verbatim}[commandchars=\\\{\}]
\PY{k+kn}{from} \PY{n+nn}{sklearn}\PY{n+nn}{.}\PY{n+nn}{preprocessing} \PY{k+kn}{import} \PY{n}{OrdinalEncoder}

\PY{n}{housing\PYZus{}cat} \PY{o}{=} \PY{n}{housing}\PY{p}{[}\PY{p}{[}\PY{l+s+s2}{\PYZdq{}}\PY{l+s+s2}{ocean\PYZus{}proximity}\PY{l+s+s2}{\PYZdq{}}\PY{p}{]}\PY{p}{]}
\PY{n}{ordinal\PYZus{}encoder} \PY{o}{=} \PY{n}{OrdinalEncoder}\PY{p}{(}\PY{p}{)}
\PY{n}{housing\PYZus{}cat\PYZus{}encoded} \PY{o}{=} \PY{n}{ordinal\PYZus{}encoder}\PY{o}{.}\PY{n}{fit\PYZus{}transform}\PY{p}{(}\PY{n}{housing\PYZus{}cat}\PY{p}{)}
\PY{n}{housing\PYZus{}cat\PYZus{}encoded}\PY{p}{[}\PY{p}{:}\PY{l+m+mi}{10}\PY{p}{]}
\end{Verbatim}
\end{tcolorbox}

            \begin{tcolorbox}[breakable, size=fbox, boxrule=.5pt, pad at break*=1mm, opacityfill=0]
\prompt{Out}{outcolor}{25}{\boxspacing}
\begin{Verbatim}[commandchars=\\\{\}]
array([[0.],
       [0.],
       [4.],
       [1.],
       [0.],
       [1.],
       [0.],
       [1.],
       [0.],
       [0.]])
\end{Verbatim}
\end{tcolorbox}
        
    Another solutions could be use one hot encoding, which create one binary
attribute per category, avoiding the issue of distances with these
non-ordinal attribute. Scikit-Learn also provides a OneHotEncoder class
to convert categorical values into one-hot vectors

    \begin{tcolorbox}[breakable, size=fbox, boxrule=1pt, pad at break*=1mm,colback=cellbackground, colframe=cellborder]
\prompt{In}{incolor}{26}{\boxspacing}
\begin{Verbatim}[commandchars=\\\{\}]
\PY{k+kn}{from} \PY{n+nn}{sklearn}\PY{n+nn}{.}\PY{n+nn}{preprocessing} \PY{k+kn}{import} \PY{n}{OneHotEncoder}

\PY{n}{cat\PYZus{}encoder} \PY{o}{=} \PY{n}{OneHotEncoder}\PY{p}{(}\PY{p}{)}
\PY{n}{housing\PYZus{}cat\PYZus{}1hot} \PY{o}{=} \PY{n}{cat\PYZus{}encoder}\PY{o}{.}\PY{n}{fit\PYZus{}transform}\PY{p}{(}\PY{n}{housing\PYZus{}cat}\PY{p}{)}
\PY{n}{housing\PYZus{}cat\PYZus{}1hot}
\end{Verbatim}
\end{tcolorbox}

            \begin{tcolorbox}[breakable, size=fbox, boxrule=.5pt, pad at break*=1mm, opacityfill=0]
\prompt{Out}{outcolor}{26}{\boxspacing}
\begin{Verbatim}[commandchars=\\\{\}]
<16512x5 sparse matrix of type '<class 'numpy.float64'>'
        with 16512 stored elements in Compressed Sparse Row format>
\end{Verbatim}
\end{tcolorbox}
        
    A really good news it it that the result is stored in a SciPy sparse
matrix instead of a NumPy array. This is very useful when you have
categorical variables with thousands of categories, after
one\_hot\_encoding we get a matrix with thousands of columns, and is
full zeros except for a single one per row. Using up tons of extra space
and memory to store zeros, so instead a sparse matrix only store the
location of non zero elements.

    \hypertarget{custom-transformers}{%
\subsection{Custom Transformers}\label{custom-transformers}}

    Although Scikit-Learn provides many useful transformers, you will need
to write your own for tasks such as custom cleanup operations or
combining specific attributes. You will want your transformer to work
seamlessly with Scikit-Learn functionalities (such as pipelines), and
since Scikit-Learn relies on duck typing (not inheritance), all you need
is to create a class and implement three methods: fit() (returning
self), transform(), and fit\_transform(). You can get the last one for
free by simply adding TransformerMixin as a base class. Also, if you add
BaseEstimator as a base class (and avoid \emph{args and }kargs in your
constructor) you will get two extra methods (get\_params() and
set\_params()) that will be useful for automatic hyperparameter tuning.
For example, here is a small transformer class that adds the combined
attributes we discussed earlier:

    \begin{tcolorbox}[breakable, size=fbox, boxrule=1pt, pad at break*=1mm,colback=cellbackground, colframe=cellborder]
\prompt{In}{incolor}{27}{\boxspacing}
\begin{Verbatim}[commandchars=\\\{\}]
\PY{k+kn}{from} \PY{n+nn}{sklearn}\PY{n+nn}{.}\PY{n+nn}{base} \PY{k+kn}{import} \PY{n}{BaseEstimator}\PY{p}{,} \PY{n}{TransformerMixin}

\PY{n}{rooms\PYZus{}ix}\PY{p}{,} \PY{n}{bedrooms\PYZus{}ix}\PY{p}{,} \PY{n}{population\PYZus{}ix}\PY{p}{,} \PY{n}{households\PYZus{}ix} \PY{o}{=} \PY{l+m+mi}{3}\PY{p}{,} \PY{l+m+mi}{4}\PY{p}{,} \PY{l+m+mi}{5}\PY{p}{,} \PY{l+m+mi}{6}

\PY{k}{class} \PY{n+nc}{CombinedAttributesAdder}\PY{p}{(}\PY{n}{BaseEstimator}\PY{p}{,} \PY{n}{TransformerMixin}\PY{p}{)}\PY{p}{:}
    \PY{k}{def} \PY{n+nf+fm}{\PYZus{}\PYZus{}init\PYZus{}\PYZus{}}\PY{p}{(}\PY{n+nb+bp}{self}\PY{p}{,} \PY{n}{add\PYZus{}bedrooms\PYZus{}per\PYZus{}room} \PY{o}{=} \PY{k+kc}{True}\PY{p}{)}\PY{p}{:} \PY{c+c1}{\PYZsh{} no *args or **kargs}
        \PY{n+nb+bp}{self}\PY{o}{.}\PY{n}{add\PYZus{}bedrooms\PYZus{}per\PYZus{}room} \PY{o}{=} \PY{n}{add\PYZus{}bedrooms\PYZus{}per\PYZus{}room}
    \PY{k}{def} \PY{n+nf}{fit}\PY{p}{(}\PY{n+nb+bp}{self}\PY{p}{,} \PY{n}{X}\PY{p}{,} \PY{n}{y}\PY{o}{=}\PY{k+kc}{None}\PY{p}{)}\PY{p}{:}
        \PY{k}{return} \PY{n+nb+bp}{self}  \PY{c+c1}{\PYZsh{} nothing else to do}
    \PY{k}{def} \PY{n+nf}{transform}\PY{p}{(}\PY{n+nb+bp}{self}\PY{p}{,} \PY{n}{X}\PY{p}{,} \PY{n}{y}\PY{o}{=}\PY{k+kc}{None}\PY{p}{)}\PY{p}{:}
        \PY{n}{rooms\PYZus{}per\PYZus{}household} \PY{o}{=} \PY{n}{X}\PY{p}{[}\PY{p}{:}\PY{p}{,} \PY{n}{rooms\PYZus{}ix}\PY{p}{]} \PY{o}{/} \PY{n}{X}\PY{p}{[}\PY{p}{:}\PY{p}{,} \PY{n}{households\PYZus{}ix}\PY{p}{]}
        \PY{n}{population\PYZus{}per\PYZus{}household} \PY{o}{=} \PY{n}{X}\PY{p}{[}\PY{p}{:}\PY{p}{,} \PY{n}{population\PYZus{}ix}\PY{p}{]} \PY{o}{/} \PY{n}{X}\PY{p}{[}\PY{p}{:}\PY{p}{,} \PY{n}{households\PYZus{}ix}\PY{p}{]}
        \PY{k}{if} \PY{n+nb+bp}{self}\PY{o}{.}\PY{n}{add\PYZus{}bedrooms\PYZus{}per\PYZus{}room}\PY{p}{:}
            \PY{n}{bedrooms\PYZus{}per\PYZus{}room} \PY{o}{=} \PY{n}{X}\PY{p}{[}\PY{p}{:}\PY{p}{,} \PY{n}{bedrooms\PYZus{}ix}\PY{p}{]} \PY{o}{/} \PY{n}{X}\PY{p}{[}\PY{p}{:}\PY{p}{,} \PY{n}{rooms\PYZus{}ix}\PY{p}{]}
            \PY{k}{return} \PY{n}{np}\PY{o}{.}\PY{n}{c\PYZus{}}\PY{p}{[}\PY{n}{X}\PY{p}{,} \PY{n}{rooms\PYZus{}per\PYZus{}household}\PY{p}{,} \PY{n}{population\PYZus{}per\PYZus{}household}\PY{p}{,}
                         \PY{n}{bedrooms\PYZus{}per\PYZus{}room}\PY{p}{]}
        \PY{k}{else}\PY{p}{:}
            \PY{k}{return} \PY{n}{np}\PY{o}{.}\PY{n}{c\PYZus{}}\PY{p}{[}\PY{n}{X}\PY{p}{,} \PY{n}{rooms\PYZus{}per\PYZus{}household}\PY{p}{,} \PY{n}{population\PYZus{}per\PYZus{}household}\PY{p}{]}

\PY{n}{attr\PYZus{}adder} \PY{o}{=} \PY{n}{CombinedAttributesAdder}\PY{p}{(}\PY{n}{add\PYZus{}bedrooms\PYZus{}per\PYZus{}room}\PY{o}{=}\PY{k+kc}{False}\PY{p}{)}
\PY{n}{housing\PYZus{}extra\PYZus{}attribs} \PY{o}{=} \PY{n}{attr\PYZus{}adder}\PY{o}{.}\PY{n}{transform}\PY{p}{(}\PY{n}{housing}\PY{o}{.}\PY{n}{values}\PY{p}{)}
\end{Verbatim}
\end{tcolorbox}

    In this example the transformer has one hyperparameter,
add\_bedrooms\_per\_room, set to True by default (it is often helpful to
provide sensible defaults). This hyperparameter will allow you to easily
find out whether adding this attribute helps the Machine Learning
algorithms or not. More generally, you can add a hyperparameter to gate
any data preparation step that you are not 100\% sure about. The more
you automate these data preparation steps, the more combinations you can
automatically try out, making it much more likely that you will find a
great combination (and saving you a lot of time).

    \hypertarget{feature-scaling}{%
\subsection{Feature Scaling}\label{feature-scaling}}

    Feature scaling it is one of the most important transfromations we need
to apply to our data, mostly ML algorithms works better with scaled
data, when the input numerical attributes have very different scales
some variables can get more influence in the results due to the scale of
the data. The scaling of the data will be implemented in the next
section.

    \hypertarget{transformation-pipelines}{%
\subsection{Transformation Pipelines}\label{transformation-pipelines}}

As you can see, there are many data transformation steps that need to be
executed in the right order. Fortunately, Scikit-Learn provides the
Pipeline class to help with such sequences of transformations. Here is a
small pipeline for the numerical attributes:

    \begin{tcolorbox}[breakable, size=fbox, boxrule=1pt, pad at break*=1mm,colback=cellbackground, colframe=cellborder]
\prompt{In}{incolor}{28}{\boxspacing}
\begin{Verbatim}[commandchars=\\\{\}]
\PY{k+kn}{from} \PY{n+nn}{sklearn}\PY{n+nn}{.}\PY{n+nn}{pipeline} \PY{k+kn}{import} \PY{n}{Pipeline}
\PY{k+kn}{from} \PY{n+nn}{sklearn}\PY{n+nn}{.}\PY{n+nn}{preprocessing} \PY{k+kn}{import} \PY{n}{StandardScaler}

\PY{n}{num\PYZus{}pipeline} \PY{o}{=} \PY{n}{Pipeline}\PY{p}{(}\PY{p}{[}
        \PY{p}{(}\PY{l+s+s1}{\PYZsq{}}\PY{l+s+s1}{imputer}\PY{l+s+s1}{\PYZsq{}}\PY{p}{,} \PY{n}{SimpleImputer}\PY{p}{(}\PY{n}{strategy}\PY{o}{=}\PY{l+s+s2}{\PYZdq{}}\PY{l+s+s2}{median}\PY{l+s+s2}{\PYZdq{}}\PY{p}{)}\PY{p}{)}\PY{p}{,}
        \PY{p}{(}\PY{l+s+s1}{\PYZsq{}}\PY{l+s+s1}{attribs\PYZus{}adder}\PY{l+s+s1}{\PYZsq{}}\PY{p}{,} \PY{n}{CombinedAttributesAdder}\PY{p}{(}\PY{p}{)}\PY{p}{)}\PY{p}{,}
        \PY{p}{(}\PY{l+s+s1}{\PYZsq{}}\PY{l+s+s1}{std\PYZus{}scaler}\PY{l+s+s1}{\PYZsq{}}\PY{p}{,} \PY{n}{StandardScaler}\PY{p}{(}\PY{p}{)}\PY{p}{)}\PY{p}{,}
    \PY{p}{]}\PY{p}{)}

\PY{n}{housing\PYZus{}num\PYZus{}tr} \PY{o}{=} \PY{n}{num\PYZus{}pipeline}\PY{o}{.}\PY{n}{fit\PYZus{}transform}\PY{p}{(}\PY{n}{housing\PYZus{}num}\PY{p}{)}
\end{Verbatim}
\end{tcolorbox}

    So far, we have handled the categorical columns and the numerical
columns separately. It would be more convenient to have a single
transformer able to handle all columns, applying the appropriate
transformations to each column. In version 0.20, Scikit-Learn introduced
the ColumnTransformer for this purpose, and the good news is that it
works great with Pandas DataFrames. Let's use it to apply all the
transformations to the housing data:

    \begin{tcolorbox}[breakable, size=fbox, boxrule=1pt, pad at break*=1mm,colback=cellbackground, colframe=cellborder]
\prompt{In}{incolor}{29}{\boxspacing}
\begin{Verbatim}[commandchars=\\\{\}]
\PY{k+kn}{from} \PY{n+nn}{sklearn}\PY{n+nn}{.}\PY{n+nn}{compose} \PY{k+kn}{import} \PY{n}{ColumnTransformer}

\PY{n}{num\PYZus{}attribs} \PY{o}{=} \PY{n+nb}{list}\PY{p}{(}\PY{n}{housing\PYZus{}num}\PY{p}{)}
\PY{n}{cat\PYZus{}attribs} \PY{o}{=} \PY{p}{[}\PY{l+s+s2}{\PYZdq{}}\PY{l+s+s2}{ocean\PYZus{}proximity}\PY{l+s+s2}{\PYZdq{}}\PY{p}{]}

\PY{n}{full\PYZus{}pipeline} \PY{o}{=} \PY{n}{ColumnTransformer}\PY{p}{(}\PY{p}{[}
        \PY{p}{(}\PY{l+s+s2}{\PYZdq{}}\PY{l+s+s2}{num}\PY{l+s+s2}{\PYZdq{}}\PY{p}{,} \PY{n}{num\PYZus{}pipeline}\PY{p}{,} \PY{n}{num\PYZus{}attribs}\PY{p}{)}\PY{p}{,}
        \PY{p}{(}\PY{l+s+s2}{\PYZdq{}}\PY{l+s+s2}{cat}\PY{l+s+s2}{\PYZdq{}}\PY{p}{,} \PY{n}{OneHotEncoder}\PY{p}{(}\PY{p}{)}\PY{p}{,} \PY{n}{cat\PYZus{}attribs}\PY{p}{)}\PY{p}{,}
    \PY{p}{]}\PY{p}{)}

\PY{n}{housing\PYZus{}prepared} \PY{o}{=} \PY{n}{full\PYZus{}pipeline}\PY{o}{.}\PY{n}{fit\PYZus{}transform}\PY{p}{(}\PY{n}{housing}\PY{p}{)}
\end{Verbatim}
\end{tcolorbox}

    Here is how this works: first we import the ColumnTransformer class,
next we get the list of numerical column names and the list of
categorical column names, and we construct a ColumnTransformer. The
constructor requires a list of tuples, where each tuple contains a
name21, a transformer and a list of names (or indices) of columns that
the transformer should be applied to. In this example, we specify that
the numerical columns should be transformed using the num\_pipeline that
we defined earlier, and the categorical columns should be transformed
using a OneHotEncoder. Finally, we apply this ColumnTransformer to the
housing data: it applies each transformer to the appropriate columns and
concatenates the outputs along the second axis (the transformers must
return the same number of rows).

    \hypertarget{select-and-train-a-model}{%
\section{Select and Train a Model}\label{select-and-train-a-model}}

In this section as final part of the project, we will select and train a
Machine Learning model.

    \hypertarget{training-and-evaluating-on-the-training-set}{%
\subsection{Training and Evaluating on the Training
Set}\label{training-and-evaluating-on-the-training-set}}

    Now things are going to be simpler than you might think, it is due to
the previous work. Let's first train a Linear Regresion model.

    \begin{tcolorbox}[breakable, size=fbox, boxrule=1pt, pad at break*=1mm,colback=cellbackground, colframe=cellborder]
\prompt{In}{incolor}{30}{\boxspacing}
\begin{Verbatim}[commandchars=\\\{\}]
\PY{k+kn}{from} \PY{n+nn}{sklearn}\PY{n+nn}{.}\PY{n+nn}{linear\PYZus{}model} \PY{k+kn}{import} \PY{n}{LinearRegression}

\PY{n}{lin\PYZus{}reg} \PY{o}{=} \PY{n}{LinearRegression}\PY{p}{(}\PY{p}{)}
\PY{n}{lin\PYZus{}reg}\PY{o}{.}\PY{n}{fit}\PY{p}{(}\PY{n}{housing\PYZus{}prepared}\PY{p}{,} \PY{n}{housing\PYZus{}labels}\PY{p}{)}
\end{Verbatim}
\end{tcolorbox}

            \begin{tcolorbox}[breakable, size=fbox, boxrule=.5pt, pad at break*=1mm, opacityfill=0]
\prompt{Out}{outcolor}{30}{\boxspacing}
\begin{Verbatim}[commandchars=\\\{\}]
LinearRegression()
\end{Verbatim}
\end{tcolorbox}
        
    You now have a working Linear Regression model. Let's try it out on a
few instances from the training set:

    \begin{tcolorbox}[breakable, size=fbox, boxrule=1pt, pad at break*=1mm,colback=cellbackground, colframe=cellborder]
\prompt{In}{incolor}{31}{\boxspacing}
\begin{Verbatim}[commandchars=\\\{\}]
\PY{n}{some\PYZus{}data} \PY{o}{=} \PY{n}{housing}\PY{o}{.}\PY{n}{iloc}\PY{p}{[}\PY{p}{:}\PY{l+m+mi}{5}\PY{p}{]}
\PY{n}{some\PYZus{}labels} \PY{o}{=} \PY{n}{housing\PYZus{}labels}\PY{o}{.}\PY{n}{iloc}\PY{p}{[}\PY{p}{:}\PY{l+m+mi}{5}\PY{p}{]}
\PY{n}{some\PYZus{}data\PYZus{}prepared} \PY{o}{=} \PY{n}{full\PYZus{}pipeline}\PY{o}{.}\PY{n}{transform}\PY{p}{(}\PY{n}{some\PYZus{}data}\PY{p}{)}
\PY{n+nb}{print}\PY{p}{(}\PY{l+s+s2}{\PYZdq{}}\PY{l+s+s2}{Predictions:}\PY{l+s+s2}{\PYZdq{}}\PY{p}{,} \PY{n}{lin\PYZus{}reg}\PY{o}{.}\PY{n}{predict}\PY{p}{(}\PY{n}{some\PYZus{}data\PYZus{}prepared}\PY{p}{)}\PY{p}{)}
\PY{n+nb}{print}\PY{p}{(}\PY{l+s+s2}{\PYZdq{}}\PY{l+s+s2}{Labels:}\PY{l+s+s2}{\PYZdq{}}\PY{p}{,} \PY{n+nb}{list}\PY{p}{(}\PY{n}{some\PYZus{}labels}\PY{p}{)}\PY{p}{)}
\end{Verbatim}
\end{tcolorbox}

    \begin{Verbatim}[commandchars=\\\{\}]
Predictions: [210644.60459286 317768.80697211 210956.43331178  59218.98886849
 189747.55849879]
Labels: [286600.0, 340600.0, 196900.0, 46300.0, 254500.0]
    \end{Verbatim}

    It works, although the predictions are not exactly accurate (e.g., the
first prediction is off by close to 40\%!). Let's measure this
regression model's RMSE on the whole training set using Scikit-Learn's
mean\_squared\_error function:

    \begin{tcolorbox}[breakable, size=fbox, boxrule=1pt, pad at break*=1mm,colback=cellbackground, colframe=cellborder]
\prompt{In}{incolor}{32}{\boxspacing}
\begin{Verbatim}[commandchars=\\\{\}]
\PY{k+kn}{from} \PY{n+nn}{sklearn}\PY{n+nn}{.}\PY{n+nn}{metrics} \PY{k+kn}{import} \PY{n}{mean\PYZus{}squared\PYZus{}error}

\PY{n}{housing\PYZus{}predictions} \PY{o}{=} \PY{n}{lin\PYZus{}reg}\PY{o}{.}\PY{n}{predict}\PY{p}{(}\PY{n}{housing\PYZus{}prepared}\PY{p}{)}
\PY{n}{lin\PYZus{}mse} \PY{o}{=} \PY{n}{mean\PYZus{}squared\PYZus{}error}\PY{p}{(}\PY{n}{housing\PYZus{}labels}\PY{p}{,} \PY{n}{housing\PYZus{}predictions}\PY{p}{)}
\PY{n}{lin\PYZus{}rmse} \PY{o}{=} \PY{n}{np}\PY{o}{.}\PY{n}{sqrt}\PY{p}{(}\PY{n}{lin\PYZus{}mse}\PY{p}{)}
\PY{n}{lin\PYZus{}rmse}
\end{Verbatim}
\end{tcolorbox}

            \begin{tcolorbox}[breakable, size=fbox, boxrule=.5pt, pad at break*=1mm, opacityfill=0]
\prompt{Out}{outcolor}{32}{\boxspacing}
\begin{Verbatim}[commandchars=\\\{\}]
68628.19819848923
\end{Verbatim}
\end{tcolorbox}
        
    Okay, this is better than nothing but clearly not a great score. This is
a example of a model underfitting the training data. The main ways to
fix underfitting are to select a more powerful model, to feed the
training algorithm with better features, or to reduce the constraints on
the model. Let's try the first options selecting a differente model.

    \hypertarget{decision-tree-regressor}{%
\subsection{Decision Tree Regressor}\label{decision-tree-regressor}}

This time we will try a DecisionTreeRegressor.

    \begin{tcolorbox}[breakable, size=fbox, boxrule=1pt, pad at break*=1mm,colback=cellbackground, colframe=cellborder]
\prompt{In}{incolor}{33}{\boxspacing}
\begin{Verbatim}[commandchars=\\\{\}]
\PY{k+kn}{from} \PY{n+nn}{sklearn}\PY{n+nn}{.}\PY{n+nn}{tree} \PY{k+kn}{import} \PY{n}{DecisionTreeRegressor}

\PY{n}{tree\PYZus{}reg} \PY{o}{=} \PY{n}{DecisionTreeRegressor}\PY{p}{(}\PY{p}{)}
\PY{n}{tree\PYZus{}reg}\PY{o}{.}\PY{n}{fit}\PY{p}{(}\PY{n}{housing\PYZus{}prepared}\PY{p}{,} \PY{n}{housing\PYZus{}labels}\PY{p}{)}
\end{Verbatim}
\end{tcolorbox}

            \begin{tcolorbox}[breakable, size=fbox, boxrule=.5pt, pad at break*=1mm, opacityfill=0]
\prompt{Out}{outcolor}{33}{\boxspacing}
\begin{Verbatim}[commandchars=\\\{\}]
DecisionTreeRegressor()
\end{Verbatim}
\end{tcolorbox}
        
    \begin{tcolorbox}[breakable, size=fbox, boxrule=1pt, pad at break*=1mm,colback=cellbackground, colframe=cellborder]
\prompt{In}{incolor}{34}{\boxspacing}
\begin{Verbatim}[commandchars=\\\{\}]
\PY{n}{housing\PYZus{}predictions} \PY{o}{=} \PY{n}{tree\PYZus{}reg}\PY{o}{.}\PY{n}{predict}\PY{p}{(}\PY{n}{housing\PYZus{}prepared}\PY{p}{)}
\PY{n}{tree\PYZus{}mse} \PY{o}{=} \PY{n}{mean\PYZus{}squared\PYZus{}error}\PY{p}{(}\PY{n}{housing\PYZus{}labels}\PY{p}{,} \PY{n}{housing\PYZus{}predictions}\PY{p}{)}
\PY{n}{tree\PYZus{}rmse} \PY{o}{=} \PY{n}{np}\PY{o}{.}\PY{n}{sqrt}\PY{p}{(}\PY{n}{tree\PYZus{}mse}\PY{p}{)}
\PY{n}{tree\PYZus{}rmse}
\end{Verbatim}
\end{tcolorbox}

            \begin{tcolorbox}[breakable, size=fbox, boxrule=.5pt, pad at break*=1mm, opacityfill=0]
\prompt{Out}{outcolor}{34}{\boxspacing}
\begin{Verbatim}[commandchars=\\\{\}]
0.0
\end{Verbatim}
\end{tcolorbox}
        
    Absolutely amazing!!! Could this model really be absolutely perfect? Of
course, it is much more likely that the model has badly overfit the
data. How can you be sure? As we saw earlier, you don't want to touch
the test set until you are ready to launch a model you are confident
about, so you need to use part of the training set for training, and
part for model validation.

    \hypertarget{evaluation-using-cross-validation}{%
\subsection{Evaluation Using
Cross-Validation}\label{evaluation-using-cross-validation}}

    Scikit-Learn's cross-validation feature it is a great tool. The
following code performs K-fold cross-validation: it randomly splits the
training set into 10 distinct subsets called folds, then it trains and
evaluates the Decision Tree model 10 times, picking a different fold for
evaluation every time and training on the other 9 folds. The result is
an array containing the 10 evaluation scores

    \begin{tcolorbox}[breakable, size=fbox, boxrule=1pt, pad at break*=1mm,colback=cellbackground, colframe=cellborder]
\prompt{In}{incolor}{35}{\boxspacing}
\begin{Verbatim}[commandchars=\\\{\}]
\PY{k+kn}{from} \PY{n+nn}{sklearn}\PY{n+nn}{.}\PY{n+nn}{model\PYZus{}selection} \PY{k+kn}{import} \PY{n}{cross\PYZus{}val\PYZus{}score}

\PY{n}{scores} \PY{o}{=} \PY{n}{cross\PYZus{}val\PYZus{}score}\PY{p}{(}\PY{n}{tree\PYZus{}reg}\PY{p}{,} \PY{n}{housing\PYZus{}prepared}\PY{p}{,} \PY{n}{housing\PYZus{}labels}\PY{p}{,}
                         \PY{n}{scoring}\PY{o}{=}\PY{l+s+s2}{\PYZdq{}}\PY{l+s+s2}{neg\PYZus{}mean\PYZus{}squared\PYZus{}error}\PY{l+s+s2}{\PYZdq{}}\PY{p}{,} \PY{n}{cv}\PY{o}{=}\PY{l+m+mi}{10}\PY{p}{)}
\PY{n}{tree\PYZus{}rmse\PYZus{}scores} \PY{o}{=} \PY{n}{np}\PY{o}{.}\PY{n}{sqrt}\PY{p}{(}\PY{o}{\PYZhy{}}\PY{n}{scores}\PY{p}{)}
\end{Verbatim}
\end{tcolorbox}

    The results are:

    \begin{tcolorbox}[breakable, size=fbox, boxrule=1pt, pad at break*=1mm,colback=cellbackground, colframe=cellborder]
\prompt{In}{incolor}{36}{\boxspacing}
\begin{Verbatim}[commandchars=\\\{\}]
\PY{k}{def} \PY{n+nf}{display\PYZus{}scores}\PY{p}{(}\PY{n}{scores}\PY{p}{)}\PY{p}{:}
    \PY{n+nb}{print}\PY{p}{(}\PY{l+s+s2}{\PYZdq{}}\PY{l+s+s2}{Scores:}\PY{l+s+s2}{\PYZdq{}}\PY{p}{,} \PY{n}{scores}\PY{p}{)}
    \PY{n+nb}{print}\PY{p}{(}\PY{l+s+s2}{\PYZdq{}}\PY{l+s+s2}{Mean:}\PY{l+s+s2}{\PYZdq{}}\PY{p}{,} \PY{n}{scores}\PY{o}{.}\PY{n}{mean}\PY{p}{(}\PY{p}{)}\PY{p}{)}
    \PY{n+nb}{print}\PY{p}{(}\PY{l+s+s2}{\PYZdq{}}\PY{l+s+s2}{Standard deviation:}\PY{l+s+s2}{\PYZdq{}}\PY{p}{,} \PY{n}{scores}\PY{o}{.}\PY{n}{std}\PY{p}{(}\PY{p}{)}\PY{p}{)}
    
\PY{n}{display\PYZus{}scores}\PY{p}{(}\PY{n}{tree\PYZus{}rmse\PYZus{}scores}\PY{p}{)}
\end{Verbatim}
\end{tcolorbox}

    \begin{Verbatim}[commandchars=\\\{\}]
Scores: [70080.5626825  67867.20975265 72334.53796405 69134.31145699
 71145.72499385 75374.1719049  70049.00668828 70738.98205283
 75387.40562744 69711.0580683 ]
Mean: 71182.29711918108
Standard deviation: 2379.6435941800673
    \end{Verbatim}

    Now this model doesn't look as good as it did earlier, in fact seems to
perform worse than the Linear Regression model. Notice that
cross-validation allows you to get not only an estimate of the
performance of your model, but also a measure of how precise this
estimate is. You would not have this information if you just used one
validation set, but at the same time cross-validation comes at the cost
of the training the model several times, so it is not always possible.

    \hypertarget{random-forest-regressor}{%
\subsection{Random Forest Regressor}\label{random-forest-regressor}}

    Now let's try one last model, RandomForestRegressor.

    \begin{tcolorbox}[breakable, size=fbox, boxrule=1pt, pad at break*=1mm,colback=cellbackground, colframe=cellborder]
\prompt{In}{incolor}{37}{\boxspacing}
\begin{Verbatim}[commandchars=\\\{\}]
\PY{k+kn}{from} \PY{n+nn}{sklearn}\PY{n+nn}{.}\PY{n+nn}{ensemble} \PY{k+kn}{import} \PY{n}{RandomForestRegressor}

\PY{n}{forest\PYZus{}reg} \PY{o}{=} \PY{n}{RandomForestRegressor}\PY{p}{(}\PY{p}{)}
\PY{n}{forest\PYZus{}reg}\PY{o}{.}\PY{n}{fit}\PY{p}{(}\PY{n}{housing\PYZus{}prepared}\PY{p}{,} \PY{n}{housing\PYZus{}labels}\PY{p}{)}

\PY{n}{housing\PYZus{}predictions} \PY{o}{=} \PY{n}{forest\PYZus{}reg}\PY{o}{.}\PY{n}{predict}\PY{p}{(}\PY{n}{housing\PYZus{}prepared}\PY{p}{)}
\PY{n}{forest\PYZus{}rmse} \PY{o}{=} \PY{n}{mean\PYZus{}squared\PYZus{}error}\PY{p}{(}\PY{n}{housing\PYZus{}labels}\PY{p}{,} \PY{n}{housing\PYZus{}predictions}\PY{p}{)}
\PY{n}{forest\PYZus{}rmse} \PY{o}{=} \PY{n}{np}\PY{o}{.}\PY{n}{sqrt}\PY{p}{(}\PY{n}{forest\PYZus{}rmse}\PY{p}{)}
\PY{n}{forest\PYZus{}rmse}
\end{Verbatim}
\end{tcolorbox}

            \begin{tcolorbox}[breakable, size=fbox, boxrule=.5pt, pad at break*=1mm, opacityfill=0]
\prompt{Out}{outcolor}{37}{\boxspacing}
\begin{Verbatim}[commandchars=\\\{\}]
18680.369219500815
\end{Verbatim}
\end{tcolorbox}
        
    This is much better: Random Forests look very promising. However, note
that the score on the training set is still much lower than on the
validation sets, meaning that the model is still overfitting the
training set.

    \hypertarget{tune-the-model}{%
\section{Tune the model}\label{tune-the-model}}

    In case that you have a shortlist of models, yo now need to fine/tune
them. In this section we will discover a few ways to to that.

    \hypertarget{grid-search}{%
\subsection{Grid Search}\label{grid-search}}

Instead of to fiddle with the hyperparameters manually you should get
Scikit-learn's GridSearchCV to search for you. With this approrach all
you need to do is telling to the model which hyperparameters you want to
play with, and what values to try uot, them it will evaluate all the
posible comninations between the values and hyperparameters, using as
before Cross-Validation.

    The folowing code searches for the best combination of hyperparameter
values for the RandomForestRegresssor.

    \begin{tcolorbox}[breakable, size=fbox, boxrule=1pt, pad at break*=1mm,colback=cellbackground, colframe=cellborder]
\prompt{In}{incolor}{38}{\boxspacing}
\begin{Verbatim}[commandchars=\\\{\}]
\PY{k+kn}{from} \PY{n+nn}{sklearn}\PY{n+nn}{.}\PY{n+nn}{model\PYZus{}selection} \PY{k+kn}{import} \PY{n}{GridSearchCV}

\PY{n}{param\PYZus{}grid} \PY{o}{=} \PY{p}{[}
    \PY{p}{\PYZob{}}\PY{l+s+s1}{\PYZsq{}}\PY{l+s+s1}{n\PYZus{}estimators}\PY{l+s+s1}{\PYZsq{}}\PY{p}{:} \PY{p}{[}\PY{l+m+mi}{3}\PY{p}{,} \PY{l+m+mi}{10}\PY{p}{,} \PY{l+m+mi}{30}\PY{p}{]}\PY{p}{,} \PY{l+s+s1}{\PYZsq{}}\PY{l+s+s1}{max\PYZus{}features}\PY{l+s+s1}{\PYZsq{}}\PY{p}{:} \PY{p}{[}\PY{l+m+mi}{2}\PY{p}{,} \PY{l+m+mi}{4}\PY{p}{,} \PY{l+m+mi}{6}\PY{p}{,} \PY{l+m+mi}{8}\PY{p}{]}\PY{p}{\PYZcb{}}\PY{p}{,}
    \PY{p}{\PYZob{}}\PY{l+s+s1}{\PYZsq{}}\PY{l+s+s1}{bootstrap}\PY{l+s+s1}{\PYZsq{}}\PY{p}{:} \PY{p}{[}\PY{k+kc}{False}\PY{p}{]}\PY{p}{,} \PY{l+s+s1}{\PYZsq{}}\PY{l+s+s1}{n\PYZus{}estimators}\PY{l+s+s1}{\PYZsq{}}\PY{p}{:} \PY{p}{[}\PY{l+m+mi}{3}\PY{p}{,} \PY{l+m+mi}{10}\PY{p}{]}\PY{p}{,} \PY{l+s+s1}{\PYZsq{}}\PY{l+s+s1}{max\PYZus{}features}\PY{l+s+s1}{\PYZsq{}}\PY{p}{:} \PY{p}{[}\PY{l+m+mi}{2}\PY{p}{,} \PY{l+m+mi}{3}\PY{p}{,} \PY{l+m+mi}{4}\PY{p}{]}\PY{p}{\PYZcb{}}\PY{p}{,}
  \PY{p}{]}

\PY{n}{forest\PYZus{}reg} \PY{o}{=} \PY{n}{RandomForestRegressor}\PY{p}{(}\PY{p}{)}

\PY{n}{grid\PYZus{}search} \PY{o}{=} \PY{n}{GridSearchCV}\PY{p}{(}\PY{n}{forest\PYZus{}reg}\PY{p}{,} \PY{n}{param\PYZus{}grid}\PY{p}{,} \PY{n}{cv}\PY{o}{=}\PY{l+m+mi}{5}\PY{p}{,}
                           \PY{n}{scoring}\PY{o}{=}\PY{l+s+s1}{\PYZsq{}}\PY{l+s+s1}{neg\PYZus{}mean\PYZus{}squared\PYZus{}error}\PY{l+s+s1}{\PYZsq{}}\PY{p}{,}
                           \PY{n}{return\PYZus{}train\PYZus{}score}\PY{o}{=}\PY{k+kc}{True}\PY{p}{)}

\PY{n}{grid\PYZus{}search}\PY{o}{.}\PY{n}{fit}\PY{p}{(}\PY{n}{housing\PYZus{}prepared}\PY{p}{,} \PY{n}{housing\PYZus{}labels}\PY{p}{)}
\end{Verbatim}
\end{tcolorbox}

            \begin{tcolorbox}[breakable, size=fbox, boxrule=.5pt, pad at break*=1mm, opacityfill=0]
\prompt{Out}{outcolor}{38}{\boxspacing}
\begin{Verbatim}[commandchars=\\\{\}]
GridSearchCV(cv=5, estimator=RandomForestRegressor(),
             param\_grid=[\{'max\_features': [2, 4, 6, 8],
                          'n\_estimators': [3, 10, 30]\},
                         \{'bootstrap': [False], 'max\_features': [2, 3, 4],
                          'n\_estimators': [3, 10]\}],
             return\_train\_score=True, scoring='neg\_mean\_squared\_error')
\end{Verbatim}
\end{tcolorbox}
        
    This param\_grid tells Scikit-Learn to first evaluate all 3 × 4 = 12
combinations of n\_estimators and max\_features hyperparameter values
specified in the first dict (don't worry about what these
hyperparameters mean for now; they will be explained in Chapter 7), then
try all 2 × 3 = 6 combinations of hyperparameter values in the second
dict, but this time with the bootstrap hyperparameter set to False
instead of True (which is the default value for this hyperparameter).

All in all, the grid search will explore 12 + 6 = 18 combinations of
RandomForestRegressor hyperparameter values, and it will train each
model five times (since we are using five-fold cross validation). In
other words, all in all, there will be 18 × 5 = 90 rounds of training!
It may take quite a long time, but when it is done you can get the best
combination of parameters like this:

    \begin{tcolorbox}[breakable, size=fbox, boxrule=1pt, pad at break*=1mm,colback=cellbackground, colframe=cellborder]
\prompt{In}{incolor}{39}{\boxspacing}
\begin{Verbatim}[commandchars=\\\{\}]
\PY{n}{grid\PYZus{}search}\PY{o}{.}\PY{n}{best\PYZus{}params\PYZus{}}
\end{Verbatim}
\end{tcolorbox}

            \begin{tcolorbox}[breakable, size=fbox, boxrule=.5pt, pad at break*=1mm, opacityfill=0]
\prompt{Out}{outcolor}{39}{\boxspacing}
\begin{Verbatim}[commandchars=\\\{\}]
\{'max\_features': 8, 'n\_estimators': 30\}
\end{Verbatim}
\end{tcolorbox}
        
    Since the results involve the maximun values that were evaluated, you
should probably try searching again woth higher values, since the score
may continue to improve, but this time we will stop here.

    We can also get the best estimator directly

    \begin{tcolorbox}[breakable, size=fbox, boxrule=1pt, pad at break*=1mm,colback=cellbackground, colframe=cellborder]
\prompt{In}{incolor}{40}{\boxspacing}
\begin{Verbatim}[commandchars=\\\{\}]
\PY{n}{grid\PYZus{}search}\PY{o}{.}\PY{n}{best\PYZus{}estimator\PYZus{}}
\end{Verbatim}
\end{tcolorbox}

            \begin{tcolorbox}[breakable, size=fbox, boxrule=.5pt, pad at break*=1mm, opacityfill=0]
\prompt{Out}{outcolor}{40}{\boxspacing}
\begin{Verbatim}[commandchars=\\\{\}]
RandomForestRegressor(max\_features=8, n\_estimators=30)
\end{Verbatim}
\end{tcolorbox}
        
    \begin{tcolorbox}[breakable, size=fbox, boxrule=1pt, pad at break*=1mm,colback=cellbackground, colframe=cellborder]
\prompt{In}{incolor}{41}{\boxspacing}
\begin{Verbatim}[commandchars=\\\{\}]
\PY{n}{cvres} \PY{o}{=} \PY{n}{grid\PYZus{}search}\PY{o}{.}\PY{n}{cv\PYZus{}results\PYZus{}}

\PY{k}{for} \PY{n}{mean\PYZus{}score}\PY{p}{,} \PY{n}{params} \PY{o+ow}{in} \PY{n+nb}{zip}\PY{p}{(}\PY{n}{cvres}\PY{p}{[}\PY{l+s+s2}{\PYZdq{}}\PY{l+s+s2}{mean\PYZus{}test\PYZus{}score}\PY{l+s+s2}{\PYZdq{}}\PY{p}{]}\PY{p}{,} \PY{n}{cvres}\PY{p}{[}\PY{l+s+s2}{\PYZdq{}}\PY{l+s+s2}{params}\PY{l+s+s2}{\PYZdq{}}\PY{p}{]}\PY{p}{)}\PY{p}{:}
    \PY{n+nb}{print}\PY{p}{(}\PY{n}{np}\PY{o}{.}\PY{n}{sqrt}\PY{p}{(}\PY{o}{\PYZhy{}}\PY{n}{mean\PYZus{}score}\PY{p}{)}\PY{p}{,} \PY{n}{params}\PY{p}{)}
\end{Verbatim}
\end{tcolorbox}

    \begin{Verbatim}[commandchars=\\\{\}]
64119.04482863937 \{'max\_features': 2, 'n\_estimators': 3\}
55262.82044462242 \{'max\_features': 2, 'n\_estimators': 10\}
53019.73319852918 \{'max\_features': 2, 'n\_estimators': 30\}
60422.14955841005 \{'max\_features': 4, 'n\_estimators': 3\}
52566.80393350787 \{'max\_features': 4, 'n\_estimators': 10\}
50369.316336569136 \{'max\_features': 4, 'n\_estimators': 30\}
59362.41640853842 \{'max\_features': 6, 'n\_estimators': 3\}
52101.52279311987 \{'max\_features': 6, 'n\_estimators': 10\}
50259.17517066402 \{'max\_features': 6, 'n\_estimators': 30\}
59987.681820357364 \{'max\_features': 8, 'n\_estimators': 3\}
52624.57943108585 \{'max\_features': 8, 'n\_estimators': 10\}
49966.51138708826 \{'max\_features': 8, 'n\_estimators': 30\}
62053.6194199234 \{'bootstrap': False, 'max\_features': 2, 'n\_estimators': 3\}
54757.9469410517 \{'bootstrap': False, 'max\_features': 2, 'n\_estimators': 10\}
59006.17929563765 \{'bootstrap': False, 'max\_features': 3, 'n\_estimators': 3\}
52730.15975808623 \{'bootstrap': False, 'max\_features': 3, 'n\_estimators': 10\}
57977.43294611441 \{'bootstrap': False, 'max\_features': 4, 'n\_estimators': 3\}
51864.4425485659 \{'bootstrap': False, 'max\_features': 4, 'n\_estimators': 10\}
    \end{Verbatim}

    \textbf{Note} : If GridSearchCV is initialized with refit=True (which is
the default), then once it finds the best estimator using
cross-validation, it retrains it on the whole training set. This is
usually a good idea since feeding it more data will likely improve its
performance.

As we can see the RSME score for the winner combination is
49847.93219931037, which is slightly better than the score we got
earlier using the default hyperparameter values (over 50k).

    \hypertarget{analyze-the-best-model-and-their-errors}{%
\subsection{Analyze the best model and their
errors}\label{analyze-the-best-model-and-their-errors}}

    You will often gain good insights on the problem by inspecting the best
models. For example, the RandomForestRegressor can indicate the relative
importance of each attribute for making accurate predictions:

    \begin{tcolorbox}[breakable, size=fbox, boxrule=1pt, pad at break*=1mm,colback=cellbackground, colframe=cellborder]
\prompt{In}{incolor}{42}{\boxspacing}
\begin{Verbatim}[commandchars=\\\{\}]
\PY{n}{feature\PYZus{}importances} \PY{o}{=} \PY{n}{grid\PYZus{}search}\PY{o}{.}\PY{n}{best\PYZus{}estimator\PYZus{}}\PY{o}{.}\PY{n}{feature\PYZus{}importances\PYZus{}}
\PY{n}{feature\PYZus{}importances}
\end{Verbatim}
\end{tcolorbox}

            \begin{tcolorbox}[breakable, size=fbox, boxrule=.5pt, pad at break*=1mm, opacityfill=0]
\prompt{Out}{outcolor}{42}{\boxspacing}
\begin{Verbatim}[commandchars=\\\{\}]
array([6.93041725e-02, 6.71490692e-02, 4.53564217e-02, 1.60844253e-02,
       1.48075969e-02, 1.52770608e-02, 1.48489683e-02, 3.82252388e-01,
       4.45655052e-02, 1.13836190e-01, 4.76698974e-02, 8.86100491e-03,
       1.53658578e-01, 1.33370279e-04, 2.40561102e-03, 3.78974052e-03])
\end{Verbatim}
\end{tcolorbox}
        
    Let's display these importance scores next to their corresponding
attribute names:

    \begin{tcolorbox}[breakable, size=fbox, boxrule=1pt, pad at break*=1mm,colback=cellbackground, colframe=cellborder]
\prompt{In}{incolor}{43}{\boxspacing}
\begin{Verbatim}[commandchars=\\\{\}]
\PY{n}{extra\PYZus{}attribs} \PY{o}{=} \PY{p}{[}\PY{l+s+s2}{\PYZdq{}}\PY{l+s+s2}{rooms\PYZus{}per\PYZus{}hhold}\PY{l+s+s2}{\PYZdq{}}\PY{p}{,} \PY{l+s+s2}{\PYZdq{}}\PY{l+s+s2}{pop\PYZus{}per\PYZus{}hhold}\PY{l+s+s2}{\PYZdq{}}\PY{p}{,} \PY{l+s+s2}{\PYZdq{}}\PY{l+s+s2}{bedrooms\PYZus{}per\PYZus{}room}\PY{l+s+s2}{\PYZdq{}}\PY{p}{]}
\PY{n}{cat\PYZus{}encoder} \PY{o}{=} \PY{n}{full\PYZus{}pipeline}\PY{o}{.}\PY{n}{named\PYZus{}transformers\PYZus{}}\PY{p}{[}\PY{l+s+s2}{\PYZdq{}}\PY{l+s+s2}{cat}\PY{l+s+s2}{\PYZdq{}}\PY{p}{]}
\PY{n}{cat\PYZus{}one\PYZus{}hot\PYZus{}attribs} \PY{o}{=} \PY{n+nb}{list}\PY{p}{(}\PY{n}{cat\PYZus{}encoder}\PY{o}{.}\PY{n}{categories\PYZus{}}\PY{p}{[}\PY{l+m+mi}{0}\PY{p}{]}\PY{p}{)}
\PY{n}{attributes} \PY{o}{=} \PY{n}{num\PYZus{}attribs} \PY{o}{+} \PY{n}{extra\PYZus{}attribs} \PY{o}{+} \PY{n}{cat\PYZus{}one\PYZus{}hot\PYZus{}attribs}
\PY{n+nb}{sorted}\PY{p}{(}\PY{n+nb}{zip}\PY{p}{(}\PY{n}{feature\PYZus{}importances}\PY{p}{,} \PY{n}{attributes}\PY{p}{)}\PY{p}{,} \PY{n}{reverse}\PY{o}{=}\PY{k+kc}{True}\PY{p}{)}
\end{Verbatim}
\end{tcolorbox}

            \begin{tcolorbox}[breakable, size=fbox, boxrule=.5pt, pad at break*=1mm, opacityfill=0]
\prompt{Out}{outcolor}{43}{\boxspacing}
\begin{Verbatim}[commandchars=\\\{\}]
[(0.38225238839183556, 'median\_income'),
 (0.1536585779590582, 'INLAND'),
 (0.11383618972364792, 'pop\_per\_hhold'),
 (0.06930417248525109, 'longitude'),
 (0.06714906923165523, 'latitude'),
 (0.04766989735748221, 'bedrooms\_per\_room'),
 (0.04535642170799949, 'housing\_median\_age'),
 (0.044565505191646024, 'rooms\_per\_hhold'),
 (0.01608442528372946, 'total\_rooms'),
 (0.015277060808083857, 'population'),
 (0.014848968276926628, 'households'),
 (0.014807596852430804, 'total\_bedrooms'),
 (0.00886100490767259, '<1H OCEAN'),
 (0.0037897405200039967, 'NEAR OCEAN'),
 (0.00240561102355084, 'NEAR BAY'),
 (0.0001333702790261435, 'ISLAND')]
\end{Verbatim}
\end{tcolorbox}
        
    With this information, you may want to try dropping some of the less
useful features (e.g., apparently only one ocean\_proximity category is
really useful, so you could try dropping the others).

You should also look at the specific errors that your system makes, then
try to understand why it makes them and what could fix the problem
(adding extra features or, on the contrary, getting rid of uninformative
ones, cleaning up outliers, etc.).

    \hypertarget{evaluate-your-system-on-the-test-set}{%
\section{Evaluate Your System on the Test
Set}\label{evaluate-your-system-on-the-test-set}}

    After tweaking your models for a while, you eventually have a system
that performs sufficiently well. Now is the time to evaluate the final
model on the test set. There is nothing special about this process; just
get the predictors and the labels from your test set, run your
full\_pipeline to transform the data (call transform(), not
fit\_transform(), you do not want to fit the test set!), and evaluate
the final model on the test set:

    \begin{tcolorbox}[breakable, size=fbox, boxrule=1pt, pad at break*=1mm,colback=cellbackground, colframe=cellborder]
\prompt{In}{incolor}{44}{\boxspacing}
\begin{Verbatim}[commandchars=\\\{\}]
\PY{n}{final\PYZus{}model} \PY{o}{=} \PY{n}{grid\PYZus{}search}\PY{o}{.}\PY{n}{best\PYZus{}estimator\PYZus{}}

\PY{n}{X\PYZus{}test} \PY{o}{=} \PY{n}{strat\PYZus{}test\PYZus{}set}\PY{o}{.}\PY{n}{drop}\PY{p}{(}\PY{l+s+s2}{\PYZdq{}}\PY{l+s+s2}{median\PYZus{}house\PYZus{}value}\PY{l+s+s2}{\PYZdq{}}\PY{p}{,} \PY{n}{axis}\PY{o}{=}\PY{l+m+mi}{1}\PY{p}{)}
\PY{n}{y\PYZus{}test} \PY{o}{=} \PY{n}{strat\PYZus{}test\PYZus{}set}\PY{p}{[}\PY{l+s+s2}{\PYZdq{}}\PY{l+s+s2}{median\PYZus{}house\PYZus{}value}\PY{l+s+s2}{\PYZdq{}}\PY{p}{]}\PY{o}{.}\PY{n}{copy}\PY{p}{(}\PY{p}{)}

\PY{n}{X\PYZus{}test\PYZus{}prepared} \PY{o}{=} \PY{n}{full\PYZus{}pipeline}\PY{o}{.}\PY{n}{transform}\PY{p}{(}\PY{n}{X\PYZus{}test}\PY{p}{)}

\PY{n}{final\PYZus{}predictions} \PY{o}{=} \PY{n}{final\PYZus{}model}\PY{o}{.}\PY{n}{predict}\PY{p}{(}\PY{n}{X\PYZus{}test\PYZus{}prepared}\PY{p}{)}

\PY{n}{final\PYZus{}mse} \PY{o}{=} \PY{n}{mean\PYZus{}squared\PYZus{}error}\PY{p}{(}\PY{n}{y\PYZus{}test}\PY{p}{,} \PY{n}{final\PYZus{}predictions}\PY{p}{)}
\PY{n}{final\PYZus{}rmse} \PY{o}{=} \PY{n}{np}\PY{o}{.}\PY{n}{sqrt}\PY{p}{(}\PY{n}{final\PYZus{}mse}\PY{p}{)}
\PY{n}{final\PYZus{}rmse}
\end{Verbatim}
\end{tcolorbox}

            \begin{tcolorbox}[breakable, size=fbox, boxrule=.5pt, pad at break*=1mm, opacityfill=0]
\prompt{Out}{outcolor}{44}{\boxspacing}
\begin{Verbatim}[commandchars=\\\{\}]
47613.458191772326
\end{Verbatim}
\end{tcolorbox}
        
    The performance will usually be slightly worse than what you measured
using cross-validation if you did a lot of hyperparameter tuning
(because your system ends up fine-tuned to perform well on the
validation data, and will likely not perform as well on unknown
datasets). It is not the case in this example, but when this happens you
must resist the temptation to tweak the hyperparameters to make the
numbers look good on the test set; the improvements would be unlikely to
generalize to new data.

    \hypertarget{future-work}{%
\section{Future work}\label{future-work}}

\begin{itemize}
\tightlist
\item
  Try a Support Vector Machine regressor (sklearn.svm.SVR)
\item
  Implement a Randomized Search in Tune the model section
\end{itemize}


    % Add a bibliography block to the postdoc
    
    
    
\end{document}
